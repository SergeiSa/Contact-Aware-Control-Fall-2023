\documentclass{beamer}

\pdfmapfile{+sansmathaccent.map}


\mode<presentation>
{
	\usetheme{Warsaw} % or try Darmstadt, Madrid, Warsaw, Rochester, CambridgeUS, ...
	\usecolortheme{seahorse} % or try seahorse, beaver, crane, wolverine, ...
	\usefonttheme{serif}  % or try serif, structurebold, ...
	\setbeamertemplate{navigation symbols}{}
	\setbeamertemplate{caption}[numbered]
} 


%%%%%%%%%%%%%%%%%%%%%%%%%%%%
% itemize settings


%%%%%%%%%%%%%%%%%%%%%%%%%%%%
% itemize settings

\definecolor{myhotpink}{RGB}{255, 80, 200}
\definecolor{mywarmpink}{RGB}{255, 60, 160}
\definecolor{mylightpink}{RGB}{255, 80, 200}
\definecolor{mypink}{RGB}{255, 30, 80}
\definecolor{mydarkpink}{RGB}{155, 25, 60}

\definecolor{mypaleblue}{RGB}{240, 240, 255}
\definecolor{mylightblue}{RGB}{120, 150, 255}
\definecolor{myblue}{RGB}{90, 90, 255}
\definecolor{mygblue}{RGB}{70, 110, 240}
\definecolor{mydarkblue}{RGB}{0, 0, 180}
\definecolor{myblackblue}{RGB}{40, 40, 120}

\definecolor{myblackturquoise}{RGB}{5, 53, 60}
\definecolor{mydarkdarkturquoise}{RGB}{8, 93, 110}
\definecolor{mydarkturquoise}{RGB}{28, 143, 150}
\definecolor{mypaleturquoise}{RGB}{230, 255, 255}
\definecolor{myturquoise}{RGB}{48, 213, 200}

\definecolor{mygreen}{RGB}{0, 200, 0}
\definecolor{mydarkgreen}{RGB}{0, 120, 0}
\definecolor{mygreen2}{RGB}{245, 255, 230}

\definecolor{mygrey}{RGB}{120, 120, 120}
\definecolor{mypalegrey}{RGB}{160, 160, 160}
\definecolor{mydarkgrey}{RGB}{80, 80, 160}

\definecolor{mydarkred}{RGB}{160, 30, 30}
\definecolor{mylightred}{RGB}{255, 150, 150}
\definecolor{myred}{RGB}{200, 110, 110}
\definecolor{myblackred}{RGB}{120, 40, 40}


\definecolor{myblackmaroon}{RGB}{50, 0, 15}

\definecolor{mygreen}{RGB}{0, 200, 0}
\definecolor{mygreen2}{RGB}{205, 255, 200}

\definecolor{mydarkcolor}{RGB}{60, 25, 155}
\definecolor{mylightcolor}{RGB}{130, 180, 250}

\setbeamertemplate{itemize items}[default]

\setbeamertemplate{itemize item}{\color{myblackmaroon}$\blacksquare$}
\setbeamertemplate{itemize subitem}{\color{mydarkdarkturquoise}$\blacktriangleright$}
\setbeamertemplate{itemize subsubitem}{\color{mygray}$\blacksquare$}

\setbeamercolor{palette quaternary}{fg=white,bg=myblackmaroon}
\setbeamercolor{titlelike}{parent=palette quaternary}

\setbeamercolor{palette quaternary2}{fg=black,bg=mypaleblue}
\setbeamercolor{frametitle}{parent=palette quaternary2}

\setbeamerfont{frametitle}{size=\Large,series=\scshape}
\setbeamerfont{framesubtitle}{size=\normalsize,series=\upshape}





%%%%%%%%%%%%%%%%%%%%%%%%%%%%
% block settings

\setbeamercolor{block title}{bg=red!30,fg=black}

\setbeamercolor*{block title example}{bg=mygreen!40!white,fg=black}

\setbeamercolor*{block body example}{fg= black, bg= mygreen2}


%%%%%%%%%%%%%%%%%%%%%%%%%%%%
% URL settings
\hypersetup{
	colorlinks=true,
	linkcolor=blue,
	filecolor=blue,      
	urlcolor=blue,
}

%%%%%%%%%%%%%%%%%%%%%%%%%%

\renewcommand{\familydefault}{\rmdefault}

\usepackage{amsmath}
\usepackage{mathtools}

\usepackage{subcaption}

\usepackage{qrcode}

\DeclareMathOperator*{\argmin}{arg\,min}
\newcommand{\bo}[1] {\mathbf{#1}}

\newcommand{\R}{\mathbb{R}} 
\newcommand{\T}{^\top}     



\newcommand{\mydate}{Fall 2023}

\newcommand{\mygit}{\textcolor{blue}{\href{https://github.com/SergeiSa/Mechatronics-2023}{github.com/SergeiSa/Mechatronics-2023}}}

\newcommand{\myqr}{ \textcolor{black}{\qrcode[height=1.5in]{https://github.com/SergeiSa/Mechatronics-2023}}
}

\newcommand{\myqrframe}{
	\begin{frame}
		\centerline{Lecture slides are available via Github, links are on Moodle}
		\bigskip
		\centerline{You can help improve these slides at:}
		\centerline{\mygit}
		\bigskip
		\myqr
	\end{frame}
}


\newcommand{\bref}[2] {\textcolor{blue}{\href{#1}{#2}}}

%%%%%%%%%%%%%%%%%%%%%%%%%%%%
% code settings

\usepackage{listings}
\usepackage{color}
% \definecolor{mygreen}{rgb}{0,0.6,0}
% \definecolor{mygray}{rgb}{0.5,0.5,0.5}
\definecolor{mymauve}{rgb}{0.58,0,0.82}
\lstset{ 
	backgroundcolor=\color{white},   % choose the background color; you must add \usepackage{color} or \usepackage{xcolor}; should come as last argument
	basicstyle=\footnotesize,        % the size of the fonts that are used for the code
	breakatwhitespace=false,         % sets if automatic breaks should only happen at whitespace
	breaklines=true,                 % sets automatic line breaking
	captionpos=b,                    % sets the caption-position to bottom
	commentstyle=\color{mygreen},    % comment style
	deletekeywords={...},            % if you want to delete keywords from the given language
	escapeinside={\%*}{*)},          % if you want to add LaTeX within your code
	extendedchars=true,              % lets you use non-ASCII characters; for 8-bits encodings only, does not work with UTF-8
	firstnumber=0000,                % start line enumeration with line 0000
	frame=single,	                   % adds a frame around the code
	keepspaces=true,                 % keeps spaces in text, useful for keeping indentation of code (possibly needs columns=flexible)
	keywordstyle=\color{blue},       % keyword style
	language=Octave,                 % the language of the code
	morekeywords={*,...},            % if you want to add more keywords to the set
	numbers=left,                    % where to put the line-numbers; possible values are (none, left, right)
	numbersep=5pt,                   % how far the line-numbers are from the code
	numberstyle=\tiny\color{mygray}, % the style that is used for the line-numbers
	rulecolor=\color{black},         % if not set, the frame-color may be changed on line-breaks within not-black text (e.g. comments (green here))
	showspaces=false,                % show spaces everywhere adding particular underscores; it overrides 'showstringspaces'
	showstringspaces=false,          % underline spaces within strings only
	showtabs=false,                  % show tabs within strings adding particular underscores
	stepnumber=2,                    % the step between two line-numbers. If it's 1, each line will be numbered
	stringstyle=\color{mymauve},     % string literal style
	tabsize=2,	                   % sets default tabsize to 2 spaces
	title=\lstname                   % show the filename of files included with \lstinputlisting; also try caption instead of title
}


%%%%%%%%%%%%%%%%%%%%%%%%%%%%
% URL settings
\hypersetup{
	colorlinks=false,
	linkcolor=blue,
	filecolor=blue,      
	urlcolor=blue,
}

%%%%%%%%%%%%%%%%%%%%%%%%%%

%%%%%%%%%%%%%%%%%%%%%%%%%%%%
% tikz settings

\usepackage{tikz}
\tikzset{every picture/.style={line width=0.75pt}}


\title{Forward Dynamics, Projectors}
\subtitle{Contact-aware Control, Lecture 3}
\author{by Sergei Savin}
\centering
\date{\mydate}



\begin{document}
\maketitle


\begin{frame}{Content}

\begin{itemize}
\item ***
\end{itemize}

\end{frame}




\begin{frame}{Manipulator DAE}
	% \framesubtitle{Parameter estimation}
	\begin{flushleft}
		
		The differential-algebraic manipulator equations:
		
		\begin{equation}
			\begin{bmatrix}
				\bo{H} & -\bo{J}\T \\
				\bo{J} & 0
			\end{bmatrix}
			\begin{bmatrix}
				\ddot{\bo{q}} \\
				\lambda
			\end{bmatrix}
			=
			\begin{bmatrix}
				\tau - \bo{C} \dot{\bo{q}} - \bo{g} \\
				-\dot{\bo{J}} \dot{\bo{q}}
			\end{bmatrix}
		\end{equation}
	 %
		have explicit solution if the Schur compliment $\bo{J}\T \bo{H}^{-1}\bo{J}$ is full rank:
		
		\begin{equation}
			\begin{bmatrix}
				\ddot{\bo{q}} \\
				\lambda
			\end{bmatrix}
		=
		\begin{bmatrix}
			\bo{H}^{-1}-\bo{H}^{-1} \bo{J}\T \bo{H}_J \bo{J} \bo{H}^{-1} &
			 \bo{H}^{-1} \bo{J}\T \bo{H}_J \\
			-\bo{H}_J \bo{J} \bo{H}^{-1} & \bo{H}_J
		\end{bmatrix}
			\begin{bmatrix}
				\tau - \bo{C} \dot{\bo{q}} - \bo{g} \\
				-\dot{\bo{J}} \dot{\bo{q}}
			\end{bmatrix}
		\end{equation}
		%
		where $\bo{H}_J = ( \bo{J} \bo{H}^{-1} \bo{J}\T )^{-1}$.
				
	\end{flushleft}
\end{frame}



\begin{frame}{Manipulator DAE solution}
	% \framesubtitle{Parameter estimation}
	\begin{flushleft}
		
		The solution can be written out block-wise:
		%
		\begin{block}{Analytic solution:}
			\begin{align}
				\ddot{\bo{q}}
				&=
				(\bo{I}-\bo{H}^{-1} \bo{J}\T \bo{H}_J \bo{J} ) \bo{H}^{-1}(\tau - \bo{C} \dot{\bo{q}} - \bo{g}) 
				-
				\bo{H}^{-1} \bo{J}\T \bo{H}_J \dot{\bo{J}} \dot{\bo{q}}
				\\
				\lambda
				&=
				-\bo{H}_J \bo{J} \bo{H}^{-1} (\tau - \bo{C} \dot{\bo{q}} - \bo{g})
				-
				\bo{H}_J \dot{\bo{J}} \dot{\bo{q}}
			\end{align}
		\end{block}
		
	\end{flushleft}
\end{frame}




\begin{frame}{Schur Projector, 1}
	% \framesubtitle{Parameter estimation}
	\begin{flushleft}
		
		Substituting expression for $\lambda$ into dynamical equation $\bo{H}\ddot{\bo{q}} - \bo{J}\T\lambda
		=
		-\bo{h}$, where $\bo{h} =  \bo{C} \dot{\bo{q}} + \bo{g} - \tau$ we get:
		
		\begin{align}
			\bo{H}\ddot{\bo{q}} + 
			\bo{J}\T\bo{H}_J \bo{J} \bo{H}^{-1} (-\bo{h}) +
			\bo{J}\T \bo{H}_J \dot{\bo{J}} \dot{\bo{q}}
			=
			(-\bo{h})
			\\
			\textcolor{myblue}
			{\bo{H}\ddot{\bo{q}} + 
			(\bo{I}-
			\bo{J}\T\bo{H}_J \bo{J} \bo{H}^{-1}) \bo{h} +
			\bo{J}\T \bo{H}_J \dot{\bo{J}} \dot{\bo{q}}
			=
			0}
		\end{align}
		
		Let us consider the following equation:
		%
		\begin{align}
			(\bo{I}-
			\bo{J}\T\bo{H}_J \bo{J} \bo{H}^{-1})(\bo{H}\ddot{\bo{q}}+\bo{h})=0
			\\
			\bo{H}\ddot{\bo{q}} - \bo{J}\T\bo{H}_J \bo{J} \ddot{\bo{q}}
			+
			(\bo{I}-
			\bo{J}\T\bo{H}_J \bo{J} \bo{H}^{-1})\bo{h} = 0
		\end{align}
		
		Since we know $\bo{J} \ddot{\bo{q}} = -\dot{\bo{J}} \dot{\bo{q}}$:
		%
		\begin{align}
			\textcolor{myblue}{
			\bo{H}\ddot{\bo{q}} + (\bo{I}-\bo{J}\T\bo{H}_J \bo{J} \bo{H}^{-1})\bo{h} + \bo{J}\T\bo{H}_J \dot{\bo{J}} \dot{\bo{q}} = 0}
		\end{align}
		
	\end{flushleft}
\end{frame}



\begin{frame}{Schur Projector, 2}
	% \framesubtitle{Parameter estimation}
	\begin{flushleft}
		

Thus, the dynamics can be re-written as: 

		\begin{align}
			\bo{P}_S(\bo{H}\ddot{\bo{q}}+\bo{h})=0
		\end{align}
	%
	where $\bo{P}_S = \bo{I}-\bo{J}\T\bo{H}_J \bo{J} \bo{H}^{-1} = 
	\bo{I}-\bo{J}\T ( \bo{J} \bo{H}^{-1} \bo{J}\T )^{-1} \bo{J} \bo{H}^{-1}$ is what we can call a Schur projector.
	
	\bigskip
	
	The projector  $\bo{P}_S$ does not change the number of equations, but it projects the dynamic onto the constraint manifold; it allows us to discard the reaction forces.
		
	\end{flushleft}
\end{frame}



\begin{frame}{Linear Algebraic Equations}
	% \framesubtitle{Parameter estimation}
	\begin{flushleft}
		
		Let us remember that a solution to any non-homogeneous (affine) linear equation $\bo{A} \bo{x} = \bo{b}$, $\bo{x} \in \R^n$ is a sum of a \emph{particular solution} and a \emph{null-space solution}:
		%
		\begin{align}
			\bo{x} = \textcolor{myblue}{\bo{A}^+\bo{b}}
			 +
			 \textcolor{mydarkgreen}{(\bo{I}- \bo{A}^+\bo{A})\bo{x}_0} 
		\end{align}
		%
		where $\bo{x}_0 \in \R^n$ is an arbitrary number.
		
		\begin{itemize}
			\item $\textcolor{myblue}{\bo{A}^+\bo{b}}$ is a particular solution. It is a smallest-norm solution to the original equation. There is one and only one particular solution, it lies in the row space of $\bo{A}$.
			
			\item $\textcolor{mydarkgreen}{(\bo{I}- \bo{A}^+\bo{A})\bo{x}_0}$ is a null space solution. There exist a $k$-dimensional space of null space solutions, where $k$ is the dimension of the null space of $\bo{A}$. If $\bo{A}\in \R^{n, n}$, its rank is $n-k$. The matrix $(\bo{I}- \bo{A}^+\bo{A})$ is a null space projector.
		\end{itemize}
		
	\end{flushleft}
\end{frame}



\begin{frame}{Manipulator DAE solution, 2}
	% \framesubtitle{Parameter estimation}
	\begin{flushleft}
		
		\textcolor{mygrey}{
		\begin{align}
			\ddot{\bo{q}}
			=
			(\bo{I}-\textcolor{myturquoise}{\bo{H}^{-1} \bo{J}\T ( \bo{J} \bo{H}^{-1} \bo{J}\T )^{-1}} \bo{J})
			 \textcolor{mylightblue}{\bo{H}^{-1}(\tau - \bo{C} \dot{\bo{q}} - \bo{g})} 
			-\\
			-
			\textcolor{myturquoise}{\bo{H}^{-1} \bo{J}\T ( \bo{J} \bo{H}^{-1} \bo{J}\T )^{-1}} \dot{\bo{J}} \dot{\bo{q}}
		\end{align}
	}
		
		Looking at the gen. acceleration solution, we recognize:
		
		\begin{itemize}
			\item Quantity $\bo{a} = \textcolor{mylightblue}{\bo{H}^{-1}(\tau - \bo{C} \dot{\bo{q}} - \bo{g})}$ is the solution (only one exists) to the unconstrained manipulator dynamics $\bo{H}\bo{a} + \bo{C} \dot{\bo{q}} + \bo{g} = \tau$.
			
			\item Quantity $\bo{J}^\# = \textcolor{myturquoise}{\bo{H}^{-1} \bo{J}\T ( \bo{J} \bo{H}^{-1} \bo{J}\T )^{-1}}$ is a weighted pseudo-inverse of $\bo{J}$ \textcolor{mygrey}{\small{(prove it for extra points)}}.
			
			\item Quantity $\bo{J}^\# \dot{\bo{J}} \dot{\bo{q}}$ is related to particular solution of the constraint equation $\bo{J} \ddot{\bo{q}} = - \dot{\bo{J}} \dot{\bo{q}}$.
		\end{itemize}
		
		
		\textcolor{mygrey}{
			\begin{align}
				\ddot{\bo{q}}
				=
				(\bo{I}-\bo{J}^\# \bo{J})
				\bo{a}
				-
				\bo{J}^\# \dot{\bo{J}} \dot{\bo{q}}
			\end{align}
		}
		
	\end{flushleft}
\end{frame}



\begin{frame}{QR projection, 1}
%	\framesubtitle{Manipulator equations}
	\begin{flushleft}
		
		For a constrained mechanical system we can solve inverse dynamics without the need for linearization. Consider the following dynamics:
		
		\begin{equation}
			\mathbf{H}\ddot{\mathbf{q}} + \bo{h} = \mathbf{J}^\top \lambda
		\end{equation}
		%
		
		where $\bo{h} =  \mathbf{C}\dot{\mathbf{q}} + \mathbf{g} - \tau$. We can represent constraint Jacobian $\mathbf{J}^\top$ as its QR decomposition: $\mathbf{J}^\top = \mathbf{Q} \begin{bmatrix} \mathbf{R} \\ \mathbf{0}  \end{bmatrix}$, where $\mathbf{Q}^\top \mathbf{Q} = \mathbf{Q} \mathbf{Q}^\top = \mathbf{I}$ and $\mathbf{R}$ is convertible.
		
		\begin{equation}
			\mathbf{H}\ddot{\mathbf{q}} + \bo{h} = \mathbf{Q} \begin{bmatrix} \mathbf{R} \\ \mathbf{0}  \end{bmatrix} \lambda
		\end{equation}
		
		
	\end{flushleft}
\end{frame}


\begin{frame}{QR projection, 2}
%	\framesubtitle{Manipulator equations, part 2}
	\begin{flushleft}
		
		Let us multiply the equation by $\mathbf{Q}^\top$:
		
		\begin{equation}
			\mathbf{Q}^\top (\mathbf{H}\ddot{\mathbf{q}} + \bo{h}) = \begin{bmatrix} \mathbf{R} \\ \mathbf{0}  \end{bmatrix} \lambda
		\end{equation}
		
		Introducing switching variables (to divide upper and lower part of the equations) $\mathbf{S}_1 = \begin{bmatrix} \mathbf{I} & \mathbf{0}  \end{bmatrix}$ and $\mathbf{S}_2 = \begin{bmatrix} \mathbf{0} & \mathbf{I}  \end{bmatrix}$ and multiplying equations by one and the other we get two systems:
		
		\begin{equation}
			\begin{cases}
				\mathbf{S}_1 \mathbf{Q}^\top (\mathbf{H}\ddot{\mathbf{q}} + \bo{h}) =\mathbf{R} \lambda 
				\\
				\mathbf{S}_2 \mathbf{Q}^\top (\mathbf{H}\ddot{\mathbf{q}} + \bo{h}) = 0
			\end{cases}
		\end{equation}
		
		The main advantage we achieved is that now we can calculate both $\mathbf{u}$ and $\lambda$
		
	\end{flushleft}
\end{frame}


\begin{frame}{QR projector}
	%	\framesubtitle{Manipulator equations, part 2}
	\begin{flushleft}
		
		Considering equation $\mathbf{S}_2 \mathbf{Q}^\top (\mathbf{H}\ddot{\mathbf{q}} + \bo{h}) = 0$ we can re-write it as:
		%
		\begin{align}
				\bo{P}_{QR} (\mathbf{H}\ddot{\mathbf{q}} + \bo{h}) = 0
		\end{align}
	%
		where $\bo{P}_{QR} = \mathbf{S}_2 \mathbf{Q}^\top$ is a QR projector. Note a similarity of the way $\bo{P}_{QR}$ and $\bo{P}_S$ act on the mechanical equations.
		
	\end{flushleft}
\end{frame}



\begin{frame}{Read more}
	
	\begin{itemize}
		
		\item Righetti, L., Buchli, J., Mistry, M. and Schaal, S., 2011, May. Inverse dynamics control of floating-base robots with external constraints: A unified view. In 2011 IEEE international conference on robotics and automation (pp. 1085-1090). IEEE. - \bref{https://drive.google.com/file/d/17lnZGk7TSr0wB47yHZYVzbWgN5kDS0RG/view}{Inverse Dynamics Control of Floating-Base Robots with External
			Constraints: a Unified View.}
		
		\item Mistry, M., Buchli, J. and Schaal, S., 2010, May. Inverse dynamics control of floating base systems using orthogonal decomposition. In 2010 IEEE international conference on robotics and automation (pp. 3406-3412). IEEE. - \bref{http://citeseerx.ist.psu.edu/viewdoc/download?doi=10.1.1.212.3601&rep=rep1&type=pdf}{citeseerx.ist.psu.edu/viewdoc/download?doi=10.1.1.212.3601}.
	\end{itemize}
	
\end{frame}



\myqrframe

\end{document}
