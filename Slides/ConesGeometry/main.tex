\documentclass{beamer}

\input{settings.tex}


\title{Geometry of  Second-Order Cones}
\subtitle{Contact-aware Control, Lecture 7}
\author{by Sergei Savin}
\centering
\date{\mydate}



\begin{document}
\maketitle


\begin{frame}{Content}

\begin{itemize}
\item 2-norm
\item Cone
\item Second-order cone (linear, affine)
\item The role of the free constant
\item Plotting level sets
\end{itemize}

\end{frame}




\begin{frame}{2-norm, 1}
	% \framesubtitle{Parameter estimation}
	\begin{flushleft}
		
		Let us consider a 2-norm as a function $f(\bo{x}): \ \R^n \rightarrow \R$:
		
		\begin{align}
			f(\bo{x}) = || \bo{x} ||_2
			\\
			f(\bo{x}) = \sqrt{ \sum_{i=1}^n x_i^2 }
		\end{align}
	
	\end{flushleft}
\end{frame}


\begin{frame}{2-norm, 2}
	% \framesubtitle{Parameter estimation}
	\begin{flushleft}
		
		We can describe 2-norm as a surface in the $\mathcal S \subset \R^{n+1}$ space:
		
		\begin{align}
			\mathcal S = \{  (J, \bo{x}): \ J = || \bo{x} ||_2 \}
		\end{align}		
		
		% TODO: \usepackage{graphicx} required
		\begin{figure}
			\centering
			\includegraphics[width=0.7\linewidth]{norm}
%			\caption{}
			\label{fig:norm}
		\end{figure}
		
	\end{flushleft}
\end{frame}



\begin{frame}{Cone}
	% \framesubtitle{Parameter estimation}
	\begin{flushleft}
		
		The shape of the surface $\mathcal S = \{  (J, \bo{x}): \ J = || \bo{x} ||_2 \}$ is a \emph{cone}. We observe the following properties of a cone:
		
		\begin{itemize}
			\item There is a single tip point $\tau$ and a normal direction.
			
			\item  Slicing cone with planes orthogonal to the normal direction, we produce ellipsoids (we can call it tangent sets).
			
			\item For any point $p$ on the cone, the half-line from the tip point $\tau$ through $p$ lies on the cone.
		\end{itemize}
	
		\bigskip
	
		The tip point for $\mathcal S$ is $(0, \bo{0})$. Normal direction is $(n, \bo{0})$. Tangent sets are circles $|| \bo{x} ||_2 = h$, parameterized by $h$. 
		
		For any point $(J(\bo{p}), \bo{p})$ we can write half-line as $\mathcal L = \{ ( J(\gamma \bo{p}), \gamma \bo{p}): \ \gamma > 0 \}$.
		
		\begin{align}
			J(\gamma \bo{p}) = ||\gamma \bo{p}|| = \gamma ||\bo{p}|| = \gamma J(\bo{p})
		\end{align}
		
	\end{flushleft}
\end{frame}



\begin{frame}{Second-order cone}
	% \framesubtitle{Parameter estimation}
	\begin{flushleft}
		
		A second-order cone constraint has the following form:
		
		\begin{equation}
			||\bo{A}\bo{x}+\bo{b}|| \leq \bo{c}\T \bo{x} + d
		\end{equation}
		%
		where $\bo{A} \in \R^{n, n}$, $\bo{b}, \bo{c} \in \R^{n}$ and $d \in \R$.
		
		\bigskip
		
		This constraint describes interior of a cone. We can prove that surface of this set is is a cone described by the following equality:
		
		\begin{equation}
			||\bo{A}\bo{x}+\bo{b}|| = \bo{c}\T \bo{x} + d
		\end{equation}
	
	\end{flushleft}
\end{frame}



\begin{frame}{Second-order cone, linear 1}
	% \framesubtitle{Parameter estimation}
	\begin{flushleft}
		
		Let us consider the following simplified case:
		
		\begin{equation}
			\label{eq:Ac_cone}
			||\bo{A}\bo{x}|| = \bo{c}\T \bo{x}
		\end{equation}
		
		The tip of this cone is the point $\bo{x} = 0$, and the normal direction is $\bo{c}$.
		
		\bigskip
		
		Let us consider a point $\bo{p}$ that lies on the surface \eqref{eq:Ac_cone}: $||\bo{A}\bo{p}|| - \bo{c}\T \bo{p} = 0$. Clearly, a point $\gamma \bo{p}$ for $\gamma \geq 0$ lies on the same surface:
		
		\begin{align*}
			e = ||\bo{A}(\gamma\bo{p})|| - \bo{c}\T (\gamma \bo{p}) = 
			\gamma||\bo{A}\bo{p}|| - \gamma \bo{c}\T \bo{p} = \\
			= \gamma(||\bo{A}\bo{p}|| -\bo{c}\T \bo{p}) =0
		\end{align*}
		
		Therefore, the half-line going from tip $\bo{x} = 0$ through $\bo{p}$ lies on the cone. 
		
	\end{flushleft}
\end{frame}


\begin{frame}{Second-order cone, linear 2}
	% \framesubtitle{Parameter estimation}
	\begin{flushleft}
		
		We can consider plane $\mathcal P$ perpendicular to $\bo{c}$, distance $h / ||\bo{c}||$ away from the tip. All points on the plane are described as follows:
		
		\begin{align}
			\bo{x} = \frac{h}{||\bo{c}||^2} \bo{c} + \bo{T} \bo{z}, \ \ \forall \bo{z}
		\end{align}
		%
		where $\bo{T} = \text{null}(\bo{c}\T)$, so $\bo{c}\T\bo{T} = 0$. Then conic surface becomes:
		
		\begin{align}
			\left|\left|\bo{A}\frac{h}{||\bo{c}||^2} \bo{c} + \bo{A} \bo{T} \bo{z}\right|\right| = \bo{c}\T \left(\frac{h}{||\bo{c}||^2} \bo{c} + \bo{T} \bo{z} \right)
			\\ 
			||\bo{b}_0 + \bo{A} \bo{T} \bo{z}|| = h
		\end{align}
	%
	where $\bo{b}_0 = \frac{h}{||\bo{c}||^2} \bo{A} \bo{c}$. This is an ellipsoid. $\qed$
		
	\end{flushleft}
\end{frame}



\begin{frame}{Second-order cone, affine, 1}
	% \framesubtitle{Parameter estimation}
	\begin{flushleft}
		
		For a full second-order cone (SOC):
		%
		\begin{equation}
			||\bo{A}\bo{x}+\bo{b}|| = \bo{c}\T \bo{x} + d
		\end{equation}
	
		we can find a tip point; it corresponds to both right-hand side and left-hand side becoming zero:
		
		\begin{equation}
			\begin{cases}
				\bo{A}\bo{x}+\bo{b} = 0 \\
				\bo{c}\T \bo{x} + d = 0
			\end{cases}
		\end{equation}
		
		Given full rank matrix $\bo{A}$, the solution is $\bo{x} = -\bo{A}^{-1}\bo{b}$. The system would hold if:
		
		\begin{equation}
			-\bo{c}\T \bo{A}^{-1}\bo{b} + d = 0
		\end{equation}
		
		
	\end{flushleft}
\end{frame}



\begin{frame}{Second-order cone, affine, 2}
	% \framesubtitle{Parameter estimation}
	\begin{flushleft}
		
		Consider a point $\bo{p}_2$ on the surface $||\bo{A}\bo{p}_2+\bo{b}|| = \bo{c}\T \bo{p}_2 + d$. Let us consider an interval from tip $\bo{p}_1 = -\bo{A}^{-1}\bo{b}$ to $\bo{p}_2$:
		%
		\begin{equation}
			\bo{x} = \gamma \bo{p}_1 + (1 - \gamma) \bo{p}_2, \ \ \gamma \geq 0
		\end{equation}
		
		We can check if the points on the interval lie on the surface:
		%
		\begin{align*}
			e = ||\bo{A}\gamma \bo{p}_1+\bo{b} + \bo{A}(1 - \gamma) \bo{p}_2|| - (\gamma  \bo{c}\T\bo{p}_1 + d + (1 - \gamma)  \bo{c}\T\bo{p}_2)
		\end{align*}
	%
		\begin{align*}
			e = ||\gamma\bo{A} \bo{p}_1+\gamma\bo{b} + (1-\gamma)\bo{b} + \bo{A}(1 - \gamma) \bo{p}_2|| - \\ - (\gamma  \bo{c}\T\bo{p}_1 + \gamma d + (1-\gamma)d + (1 - \gamma)  \bo{c}\T\bo{p}_2)
		\end{align*}
		
		We know that $\bo{A}\gamma \bo{p}_1+\bo{b} = 0$ and $\bo{c}\T\bo{p}_1 + d = 0$:
		%
		\begin{align}
			e = || (1-\gamma)\bo{b} + \bo{A}(1 - \gamma) \bo{p}_2|| - (1-\gamma)d - (1 - \gamma)  \bo{c}\T\bo{p}_2
			\\
			e = || \bo{b} + \bo{A}\bo{p}_2|| - (d +\bo{c}\T\bo{p}_2) = 0
		\end{align}
		
		Thus, the interval (and therefore, the half-line on which it lies) lies on the surface.
		
	\end{flushleft}
\end{frame}



\begin{frame}{Second-order cone, affine, 3}
	% \framesubtitle{Parameter estimation}
	\begin{flushleft}
		
		Let us consider plane $\mathcal P$ perpendicular to $\bo{c}$, distance  $h / ||\bo{c}||$ away from the tip. All points on the plane are described as follows:
		
		\begin{align}
			\bo{x} = \frac{h}{||\bo{c}||^2} \bo{c} + \bo{T} \bo{z}, \ \ \forall \bo{z}
		\end{align}
		
		We can substitute it into the second-order cone:
		
		\begin{align}
			||\bo{A}(\frac{h}{||\bo{c}||^2} \bo{c} + \bo{T} \bo{z})+\bo{b}|| = \bo{c}\T (\frac{h}{||\bo{c}||^2} \bo{c} + \bo{T} \bo{z}) + d
			\\
			||\bo{b}_0+\bo{A}\bo{T} \bo{z}|| = h + d
		\end{align}
		%
			where $\bo{b}_0 = \frac{h}{||\bo{c}||^2} \bo{A} \bo{c}+\bo{b}$. This is an ellipsoid. $\qed$
	
		
	\end{flushleft}
\end{frame}




\begin{frame}{The role of the free constant, 1}
	% \framesubtitle{Parameter estimation}
	\begin{flushleft}
		
		Free constant $d$ plays a specific role in the SOC. As we said before, there is a condition $d = \bo{c}\T \bo{A}^{-1}\bo{b} + d$. We can explain it by looking at COS as intersection of two surfaces: 
		
		\begin{align}
			J = ||\bo{A}\bo{x}+\bo{b}|| \\
			P = \bo{c}\T \bo{x} + d
		\end{align}
		
		First is a cone and second is a plane. Their intersection is called a \emph{conic section}. 
				
	\end{flushleft}
\end{frame}




\begin{frame}{The role of the free constant, 2}
	% \framesubtitle{Parameter estimation}
	\begin{flushleft}
		
		Typical conic sections are shown below:
		
		% TODO: \usepackage{graphicx} required
		\begin{figure}
			\centering
			\includegraphics[width=0.7\linewidth]{cones}
			\label{fig:cones}
		\end{figure}
	
		As we can see, they represent ellipsoid and parabola. In order for them to represent a cone, the plane  $S$ needs to pass through the tip of the cone $J$. This can be achieved with the appropriate choice of constant $d$, which shifts $S$ up or down.
		
	\end{flushleft}
\end{frame}



\begin{frame}{Plotting level sets, 1}
	% \framesubtitle{Parameter estimation}
	\begin{flushleft}
		
		To plot a cone it is convenient to first use change of coordinates $\bo{y} = \bo{A}\bo{x}+\bo{b}$, meaning $\bo{x} =\bo{A}^{-1} (\bo{y} - \bo{b})$, giving us SOC:
		%
		\begin{equation}
			||\bo{y}|| = \bo{c}\T \bo{A}^{-1} (\bo{y} - \bo{b}) + d
		\end{equation}
		
		Note that $d - \bo{c}\T \bo{A}^{-1}\bo{b} = 0$ for a cone with a tip; so SOC becomes:
		%
		\begin{equation}
			||\bo{y}|| = \bo{c}\T \bo{A}^{-1} \bo{y}
		\end{equation}
		
		To plot level sets of this cone we choose height of the level set $h$ and pick point $\bo{y}_h = h \frac{\bo{A}^{-T} \bo{c}}{\bo{c}\T \bo{A}^{-1}\bo{A}^{-T} \bo{c}}$; we note that $\bo{c}\T \bo{A}^{-1} \bo{y}_h = h$. Then we consider points on the plane $\mathcal P$ orthogonal to $\bo{c}\T \bo{A}^{-1}$ and passing through $\bo{y}_h$:
		
		\begin{equation}
			\mathcal P = {\bo{y}_h + \bo{T}\bo{z}: \  \forall \bo{z}}
		\end{equation}
		 %
		 where $\bo{T} = \text{null}(\bo{c}\T \bo{A}^{-1})$, so $\bo{c}\T \bo{A}^{-1} \bo{T} = 0$.

	\end{flushleft}
\end{frame}


\begin{frame}{Plotting level sets, 2}
	% \framesubtitle{Parameter estimation}
	\begin{flushleft}
		
		Since  SOC becomes:
		%
		\begin{equation}
			||\bo{y}_h + \bo{T}\bo{z}|| = h
		\end{equation}
	
		Since $\bo{y}_h$ and $\bo{T}\bo{z}$ are orthogonal, it is equivalent to:
		%
		\begin{equation}
			||\bo{T}\bo{z}|| = g
		\end{equation}
	%
	where $g = \sqrt{ h^2 - \bo{y}_h\T\bo{y}_h }$. In the 3D case, this is a circle with radius $g$. We can find $N$ consecutive evenly spaced points of this circle, resulting in the next sequence of $\bo{y}_l$:
	
		\begin{align}
			\bo{y}_l &= \bo{y}_h + \bo{T}
			\begin{bmatrix}
				g\cos(\varphi) \\ -g\sin(\varphi)
			\end{bmatrix}, \ \ \ \varphi = 0, \ \frac{2\pi}{N},\ 2\frac{2\pi}{N}, \ ..., \ 2\pi 
		\\
		\bo{x}_l  &=\bo{A}^{-1} (\bo{y}_l - \bo{b})
		\end{align}
	
		The center of the ellipsoid representing this level set lies at the point $\bo{x} =\bo{A}^{-1} (\bo{y}_h - \bo{b})$.
		
	\end{flushleft}
\end{frame}



\begin{frame}{Plotting level sets, 3}
	% \framesubtitle{Parameter estimation}
	\begin{flushleft}
		
		
		\begin{figure}
			\centering
			\begin{subfigure}[b]{0.45\textwidth}
				\centering
				\includegraphics[width=\textwidth]{plotted1}
			\end{subfigure}
			\hfill
			\begin{subfigure}[b]{0.45\textwidth}
				\centering
				\includegraphics[width=\textwidth]{plotted2}
			\end{subfigure}
			\caption{Cone. Dashed line - centers of level-sets.}
		\end{figure}
		
	\end{flushleft}
\end{frame}




%\begin{frame}{Read more}
%	
%	\begin{itemize}
%		
%		\item ***
%		 
%		
%	\end{itemize}
%	
%\end{frame}



\myqrframe

\end{document}
