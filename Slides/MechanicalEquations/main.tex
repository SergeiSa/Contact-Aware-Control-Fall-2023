\documentclass{beamer}

\input{settings.tex}


\title{Mechanical Equations with Constraints}
\subtitle{Contact-aware Control, Lecture 2}
\author{by Sergei Savin}
\centering
\date{\mydate}



\begin{document}
\maketitle


\begin{frame}{Content}

\begin{itemize}
\item Lagrange equations
\item Generalized forces
\item Lagrange equations with constraints
\item Constraints
\item Manipulator equations
\item Constraints differentiation
\item Solution to DAE
\end{itemize}

\end{frame}




\begin{frame}{Lagrange equations}
% \framesubtitle{Parameter estimation}
\begin{flushleft}

Lagrange equations have form:

\begin{equation}
	\frac{d}{dt} \bigg( 
	\frac{\partial T }{\partial \dot{\bo{q}}}
	\bigg) - 
	\frac{\partial T }{\partial \bo{q}} = \tau
\end{equation}
%
where $T$ is kinetic energy, $\bo{q}$ is a vector of generalized coordinates and $\tau$ are generalized torques.

\bigskip

Note that kinetic energy can be described as $T = \frac{1}{2} \dot{\bo{q}}\T \bo{H} \dot{\bo{q}}$, where $\bo{H}$ is generalized inertia matrix. Matrix $\bo{H}$ is symmetric, positive-definite and full rank.

\end{flushleft}
\end{frame}


\begin{frame}{Generalized forces, 1}
	% \framesubtitle{Parameter estimation}
	\begin{flushleft}
		
		Generalized forces $\tau$ can be generated by Cartesian forces or Cartesian torques. We can describe relations between Cartesian force $\bo{f}$ and associated generalized force:
		
		\begin{equation}
			\tau_i = \left( \frac{\partial \bo{r}_i}{\partial  \bo{q}} \right)\T \bo{f}_i
		\end{equation}
		%
		where $\bo{r}_i = \bo{r}_i(\bo{q})$ is the vector describing position of the point of application of the force $\bo{f}_i$, as a function of generalized coordinates $\bo{q}$.
		
		\bigskip
		
		If we define jacobian $\bo{J}_i = \frac{\partial \bo{r}_i}{\partial  \bo{q}}$ we can re-write the relation above as:
		%
		\begin{equation}
			\tau_i = (\bo{J}^r_i)\T \bo{f}_i
		\end{equation}
	
	\end{flushleft}
\end{frame}


\begin{frame}{Generalized forces, 2}
	% \framesubtitle{Parameter estimation}
	\begin{flushleft}
		
		We can describe relations between Cartesian torque $\bo{m}$ and associated generalized force:
		
		\begin{equation}
			\tau_i = \left( \frac{\partial \omega_i}{\partial  \dot{\bo{q}}} \right)\T \bo{m}_i
		\end{equation}
		%
		where $\omega_i = \omega_i(\bo{q}, \dot{\bo{q}})$ is the angular velocity of the body to which the Cartesian torque $\bo{m}$  is applied.
		
		\bigskip
		
		If we define jacobian $\bo{J}^\omega_i = \frac{\partial \omega_i}{\partial  \dot{\bo{q}}}$ we can re-write the relation above as:
		%
		\begin{equation}
			\tau_i = (\bo{J}^\omega_i)\T \bo{m}_i
		\end{equation}
		
	\end{flushleft}
\end{frame}



\begin{frame}{Generalized forces, 3}
	% \framesubtitle{Parameter estimation}
	\begin{flushleft}
		
		We can define a \emph{wrench} $\bo{w}$:
		
		\begin{equation}
			\bo{w} = \begin{bmatrix}
				\bo{f} \\ \bo{m}
			\end{bmatrix}
		\end{equation}
		
		We can describe relations between a wrench $\bo{w}$ and associated generalized force:
		
		\begin{equation}
			\tau_i = \begin{bmatrix}
				\bo{J}^r_i \\ \bo{J}^\omega_i
			\end{bmatrix}\T
		\bo{w}_i
		=
		\bo{J}_i\T \bo{w}_i
		\end{equation}

\bigskip

Note that the total generalized force can be computed as:

\begin{equation}
	\tau = \sum \bo{J}_i\T \bo{w}_i
\end{equation}
		
	\end{flushleft}
\end{frame}


\begin{frame}{Lagrange equations with constraints}
	% \framesubtitle{Parameter estimation}
	\begin{flushleft}
		
		Lagrange equations with constraints have form:
		
		\begin{equation}
			\begin{cases}
				\frac{d}{dt} \bigg( 
				\frac{\partial T }{\partial \dot{\bo{q}}}
				\bigg) - 
				\frac{\partial T }{\partial \bo{q}} = \tau + \left( \frac{\partial \bo{r}}{\partial  \bo{q}} \right)\T \lambda
				\\
				\bo{r}(\bo{q}) = 0
			\end{cases}
		\end{equation}
		%
		where $\bo{r}(\bo{q}) = 0$ are constraints and $\lambda$ are reaction forces.

		\bigskip
		
		We can think of $\lambda$ as concatenation of all reaction forces associated with constraints.
		
	\end{flushleft}
\end{frame}



\begin{frame}{Constraints, Example}
	% \framesubtitle{Parameter estimation}
	\begin{flushleft}
		
		Let us consider a planar three link mechanism, whose end-effector is described as:
		%
		\begin{equation}
			\begin{cases}
				 x_e = l_1 \cos q_1 + l_2 \cos q_2 + l_3 \cos q_3 
				 \\
				 y_e = l_1 \sin q_1 + l_2 \sin q_2 + l_3 \sin q_3 
			\end{cases}
		\end{equation}
		
		Then, if the end-effector is attached to the ground at a point $x_e^* = 1$, $y_e^* = 0$, the constraints look like the following:
		
		\begin{equation}
	\begin{cases}
		l_1 \cos q_1 + l_2 \cos q_2 + l_3 \cos q_3 - 1 = 0
		\\
		l_1 \sin q_1 + l_2 \sin q_2 + l_3 \sin q_3 = 0
	\end{cases}
\end{equation}
		
		
	\end{flushleft}
\end{frame}



\begin{frame}{Constraints}
	% \framesubtitle{Parameter estimation}
	\begin{flushleft}
		
				In general, we distinguish between constraints \emph{expression} and constraints \emph{value}. For example, a point K on the end-effector is described by radius-vector $\bo{r}_K=\bo{r}_K(\bo{q})$. If we affix K at a particular value $\bo{r}_K^*$:
				
				\begin{itemize}
					\item $\bo{r}_K(\bo{q})$ is the constraint expression;
					
					\item  $\bo{r}_K^*$ is the constraint value.
				\end{itemize}
			
			\bigskip
			
			The constraint will take a form $\bo{r}_K(\bo{q}) - \bo{r}_K^* = 0$. Note that constraint value does not influence the constraint jacobians; therefore, as long as we only need constraint jacobians, we do not need to know constraint values.
		
	\end{flushleft}
\end{frame}



\begin{frame}{Manipulator equations}
	% \framesubtitle{Parameter estimation}
	\begin{flushleft}
		
		Manipulator equations have form:
		%
		\begin{equation}
			\bo{H} \ddot{\bo{q}} + \bo{C} \dot{\bo{q}} + \bo{g} = \tau
		\end{equation}
		%
		where $\bo{H}=\bo{H}(\bo{q})$ is generalized inertia matrix, $\bo{C}\dot{\bo{q}}$ is generalized inertial forces and $\bo{g}=\bo{g}(\bo{q})$ are generalized gravitational forces.
		
	\end{flushleft}
\end{frame}



\begin{frame}{Manipulator equations with constraints}
	% \framesubtitle{Parameter estimation}
	\begin{flushleft}
		
		Manipulator equations have form:
		%
		\begin{equation}
			\begin{cases}
				\bo{H} \ddot{\bo{q}} + \bo{C} \dot{\bo{q}} + \bo{g} = \tau + \left( \frac{\partial \bo{r}}{\partial  \bo{q}} \right)\T \lambda
				\\
				\bo{r}(\bo{q}) = 0
			\end{cases}
		\end{equation}

		Defining constraint jacobian $\bo{J} = \frac{\partial \bo{r}}{\partial  \bo{q}}$ we can re-write the equations:
		
		\begin{equation}
	\begin{cases}
		\bo{H} \ddot{\bo{q}} + \bo{C} \dot{\bo{q}} + \bo{g} = \tau + \bo{J}\T \lambda
		\\
		\bo{r}(\bo{q}) = 0
	\end{cases}
\end{equation}
		
	\end{flushleft}
\end{frame}





\begin{frame}{Constraint differentiation}
	% \framesubtitle{Parameter estimation}
	\begin{flushleft}
		
		Differentiating constraint $\bo{r}(\bo{q})$ we get:
		
		\begin{equation}
			\frac{\partial \bo{r}}{\partial \bo{q}} \frac{\partial \bo{q}}{\partial t} = 0
		\end{equation}
	
		This can be written as:
		
		\begin{equation}
			\bo{J} \dot{\bo{q}} = 0
		\end{equation}
		
		Differentiating this once more we get:
		
		\begin{equation}
			\bo{J} \ddot{\bo{q}} + \dot{\bo{J}} \dot{\bo{q}}= 0
		\end{equation}
		
	\end{flushleft}
\end{frame}



\begin{frame}{Second-order constraints}
	% \framesubtitle{Parameter estimation}
	\begin{flushleft}
		
		We can replace constraints with their second derivatives:
		%
		\begin{equation}
			\begin{cases}
				\bo{H} \ddot{\bo{q}} + \bo{C} \dot{\bo{q}} + \bo{g} = \tau + \bo{J}\T \lambda
				\\
				\bo{J} \ddot{\bo{q}} + \dot{\bo{J}} \dot{\bo{q}}= 0
			\end{cases}
		\end{equation}
	
		This is a DAE in variables $\bo{q}$ and $\lambda$. We can re-write it as a vector-matrix form:
		
		\begin{equation}
		\begin{bmatrix}
		\bo{H} & -\bo{J}\T \\
		\bo{J} & 0
	\end{bmatrix}
\begin{bmatrix}
\ddot{\bo{q}} \\
\lambda
\end{bmatrix}
=
\begin{bmatrix}
	\tau - \bo{C} \dot{\bo{q}} - \bo{g} \\
	-\dot{\bo{J}} \dot{\bo{q}}
\end{bmatrix}
		\end{equation}
		
	\end{flushleft}
\end{frame}




\begin{frame}{Solution to the DAE}
	% \framesubtitle{Parameter estimation}
	\begin{flushleft}
		
		\textcolor{mygrey}{
		\begin{equation}
			\begin{bmatrix}
				\bo{H} & -\bo{J}\T \\
				\bo{J} & 0
			\end{bmatrix}
			\begin{bmatrix}
				\ddot{\bo{q}} \\
				\lambda
			\end{bmatrix}
			=
			\begin{bmatrix}
				\tau - \bo{C} \dot{\bo{q}} - \bo{g} \\
				-\dot{\bo{J}} \dot{\bo{q}}
			\end{bmatrix}
		\end{equation}
	}
		
		The matrix-vector equation above can be solved given the following condition: the Shur compliment $(\bo{J}\bo{H}^{-1}\bo{J}\T)$ needs to be full-rank.
		
		\bigskip
		
		Given that $\bo{H}$ is positive-definite, $\bo{H}^{-1}$ is also positive-definite. Therefore $\bo{J}\bo{H}^{-1}\bo{J}\T$ is symmetric and positive-semidefinite.
		
				\bigskip
				
		For $\bo{J}\bo{H}^{-1}\bo{J}\T$ to be positive-definite (full rank), jacobian $\bo{J}$ has to be full row-rank.
		
				
	\end{flushleft}
\end{frame}




\begin{frame}{Constraint jacobian rank}
	% \framesubtitle{Parameter estimation}
	\begin{flushleft}
		
		The row rank of the constraint jacobian $\bo{J}$ depends on linear independence of constraints. Constraints are not linearly independent, we call the system \emph{overconstrained} or \emph{overdetermined}.
		
		\bigskip
		
		If constraint jacobian $\bo{J}$ has linearly dependent rows, we can define a new jacobian $\bo{J}_o = \text{row}(\bo{J})$ as an orthonormal basis in the row space of the original one, giving us relation $\bo{J} = \bo{T} \bo{J}_o$. We can then define $\gamma = \bo{T}\T \lambda$ and re-write the dynamics as:
		
		\begin{equation}
			\begin{bmatrix}
				\bo{H} & -\bo{J}_o\T \\
				\bo{J}_o & 0
			\end{bmatrix}
			\begin{bmatrix}
				\ddot{\bo{q}} \\
				\gamma
			\end{bmatrix}
			=
			\begin{bmatrix}
				\tau - \bo{C} \dot{\bo{q}} - \bo{g} \\
				-\dot{\bo{J}_o} \dot{\bo{q}}
			\end{bmatrix}
		\end{equation}
		
		This will not let us recover $\lambda$, but we can find $\ddot{\bo{q}}$.
		
	\end{flushleft}
\end{frame}






\myqrframe

\end{document}
