\documentclass{beamer}

\pdfmapfile{+sansmathaccent.map}


\mode<presentation>
{
	\usetheme{Warsaw} % or try Darmstadt, Madrid, Warsaw, Rochester, CambridgeUS, ...
	\usecolortheme{seahorse} % or try seahorse, beaver, crane, wolverine, ...
	\usefonttheme{serif}  % or try serif, structurebold, ...
	\setbeamertemplate{navigation symbols}{}
	\setbeamertemplate{caption}[numbered]
} 


%%%%%%%%%%%%%%%%%%%%%%%%%%%%
% itemize settings


%%%%%%%%%%%%%%%%%%%%%%%%%%%%
% itemize settings

\definecolor{myhotpink}{RGB}{255, 80, 200}
\definecolor{mywarmpink}{RGB}{255, 60, 160}
\definecolor{mylightpink}{RGB}{255, 80, 200}
\definecolor{mypink}{RGB}{255, 30, 80}
\definecolor{mydarkpink}{RGB}{155, 25, 60}

\definecolor{mypaleblue}{RGB}{240, 240, 255}
\definecolor{mylightblue}{RGB}{120, 150, 255}
\definecolor{myblue}{RGB}{90, 90, 255}
\definecolor{mygblue}{RGB}{70, 110, 240}
\definecolor{mydarkblue}{RGB}{0, 0, 180}
\definecolor{myblackblue}{RGB}{40, 40, 120}

\definecolor{myblackturquoise}{RGB}{5, 53, 60}
\definecolor{mydarkdarkturquoise}{RGB}{8, 93, 110}
\definecolor{mydarkturquoise}{RGB}{28, 143, 150}
\definecolor{mypaleturquoise}{RGB}{230, 255, 255}
\definecolor{myturquoise}{RGB}{48, 213, 200}

\definecolor{mygreen}{RGB}{0, 200, 0}
\definecolor{mydarkgreen}{RGB}{0, 120, 0}
\definecolor{mygreen2}{RGB}{245, 255, 230}

\definecolor{mygrey}{RGB}{120, 120, 120}
\definecolor{mypalegrey}{RGB}{160, 160, 160}
\definecolor{mydarkgrey}{RGB}{80, 80, 160}

\definecolor{mydarkred}{RGB}{160, 30, 30}
\definecolor{mylightred}{RGB}{255, 150, 150}
\definecolor{myred}{RGB}{200, 110, 110}
\definecolor{myblackred}{RGB}{120, 40, 40}


\definecolor{myblackmaroon}{RGB}{50, 0, 15}

\definecolor{mygreen}{RGB}{0, 200, 0}
\definecolor{mygreen2}{RGB}{205, 255, 200}

\definecolor{mydarkcolor}{RGB}{60, 25, 155}
\definecolor{mylightcolor}{RGB}{130, 180, 250}

\setbeamertemplate{itemize items}[default]

\setbeamertemplate{itemize item}{\color{myblackmaroon}$\blacksquare$}
\setbeamertemplate{itemize subitem}{\color{mydarkdarkturquoise}$\blacktriangleright$}
\setbeamertemplate{itemize subsubitem}{\color{mygray}$\blacksquare$}

\setbeamercolor{palette quaternary}{fg=white,bg=myblackmaroon}
\setbeamercolor{titlelike}{parent=palette quaternary}

\setbeamercolor{palette quaternary2}{fg=black,bg=mypaleblue}
\setbeamercolor{frametitle}{parent=palette quaternary2}

\setbeamerfont{frametitle}{size=\Large,series=\scshape}
\setbeamerfont{framesubtitle}{size=\normalsize,series=\upshape}





%%%%%%%%%%%%%%%%%%%%%%%%%%%%
% block settings

\setbeamercolor{block title}{bg=red!30,fg=black}

\setbeamercolor*{block title example}{bg=mygreen!40!white,fg=black}

\setbeamercolor*{block body example}{fg= black, bg= mygreen2}


%%%%%%%%%%%%%%%%%%%%%%%%%%%%
% URL settings
\hypersetup{
	colorlinks=true,
	linkcolor=blue,
	filecolor=blue,      
	urlcolor=blue,
}

%%%%%%%%%%%%%%%%%%%%%%%%%%

\renewcommand{\familydefault}{\rmdefault}

\usepackage{amsmath}
\usepackage{mathtools}

\usepackage{subcaption}

\usepackage{qrcode}

\DeclareMathOperator*{\argmin}{arg\,min}
\newcommand{\bo}[1] {\mathbf{#1}}

\newcommand{\R}{\mathbb{R}} 
\newcommand{\T}{^\top}     



\newcommand{\mydate}{Fall 2023}

\newcommand{\mygit}{\textcolor{blue}{\href{https://github.com/SergeiSa/Mechatronics-2023}{github.com/SergeiSa/Mechatronics-2023}}}

\newcommand{\myqr}{ \textcolor{black}{\qrcode[height=1.5in]{https://github.com/SergeiSa/Mechatronics-2023}}
}

\newcommand{\myqrframe}{
	\begin{frame}
		\centerline{Lecture slides are available via Github, links are on Moodle}
		\bigskip
		\centerline{You can help improve these slides at:}
		\centerline{\mygit}
		\bigskip
		\myqr
	\end{frame}
}


\newcommand{\bref}[2] {\textcolor{blue}{\href{#1}{#2}}}

%%%%%%%%%%%%%%%%%%%%%%%%%%%%
% code settings

\usepackage{listings}
\usepackage{color}
% \definecolor{mygreen}{rgb}{0,0.6,0}
% \definecolor{mygray}{rgb}{0.5,0.5,0.5}
\definecolor{mymauve}{rgb}{0.58,0,0.82}
\lstset{ 
	backgroundcolor=\color{white},   % choose the background color; you must add \usepackage{color} or \usepackage{xcolor}; should come as last argument
	basicstyle=\footnotesize,        % the size of the fonts that are used for the code
	breakatwhitespace=false,         % sets if automatic breaks should only happen at whitespace
	breaklines=true,                 % sets automatic line breaking
	captionpos=b,                    % sets the caption-position to bottom
	commentstyle=\color{mygreen},    % comment style
	deletekeywords={...},            % if you want to delete keywords from the given language
	escapeinside={\%*}{*)},          % if you want to add LaTeX within your code
	extendedchars=true,              % lets you use non-ASCII characters; for 8-bits encodings only, does not work with UTF-8
	firstnumber=0000,                % start line enumeration with line 0000
	frame=single,	                   % adds a frame around the code
	keepspaces=true,                 % keeps spaces in text, useful for keeping indentation of code (possibly needs columns=flexible)
	keywordstyle=\color{blue},       % keyword style
	language=Octave,                 % the language of the code
	morekeywords={*,...},            % if you want to add more keywords to the set
	numbers=left,                    % where to put the line-numbers; possible values are (none, left, right)
	numbersep=5pt,                   % how far the line-numbers are from the code
	numberstyle=\tiny\color{mygray}, % the style that is used for the line-numbers
	rulecolor=\color{black},         % if not set, the frame-color may be changed on line-breaks within not-black text (e.g. comments (green here))
	showspaces=false,                % show spaces everywhere adding particular underscores; it overrides 'showstringspaces'
	showstringspaces=false,          % underline spaces within strings only
	showtabs=false,                  % show tabs within strings adding particular underscores
	stepnumber=2,                    % the step between two line-numbers. If it's 1, each line will be numbered
	stringstyle=\color{mymauve},     % string literal style
	tabsize=2,	                   % sets default tabsize to 2 spaces
	title=\lstname                   % show the filename of files included with \lstinputlisting; also try caption instead of title
}


%%%%%%%%%%%%%%%%%%%%%%%%%%%%
% URL settings
\hypersetup{
	colorlinks=false,
	linkcolor=blue,
	filecolor=blue,      
	urlcolor=blue,
}

%%%%%%%%%%%%%%%%%%%%%%%%%%

%%%%%%%%%%%%%%%%%%%%%%%%%%%%
% tikz settings

\usepackage{tikz}
\tikzset{every picture/.style={line width=0.75pt}}


\title{Linearization, Orthogonal LQR}
\subtitle{Contact-aware Control, Lecture 5}
\author{by Sergei Savin}
\centering
\date{\mydate}



\begin{document}
\maketitle


\begin{frame}{Content}

\begin{itemize}
\item LTI system with explicit constraints
\item Linear-fractional transformation
\item LTI system with implicit constraints
\item Linearization
\item Orthogonal decomposition
\item Orthogonal projection
\item Feedback control
\item LQR
\end{itemize}

\end{frame}




\begin{frame}{LTI system with explicit constraints}
	% \framesubtitle{Parameter estimation}
	\begin{flushleft}
		
		LTI system with explicit constraints (EC-LTI) can be presented in the following form:
		
		\begin{equation}
			\begin{cases}
				\dot{\bo{x}}=\bo{A}\bo{x}+\bo{B}\bo{u}+\bo{F}\lambda 
				\\
				\bo{G}\dot{\bo{x}}=0
			\end{cases}
		\end{equation}
		%
		where $\bo{G}\dot{\bo{x}}=0$ is the constraints equation.
	
	\end{flushleft}
\end{frame}


\begin{frame}{Linear-fractional transformation}
	% \framesubtitle{Parameter estimation}
	\begin{flushleft}
		
		Substituting $\dot{\bo{x}}=\bo{A}\bo{x}+\bo{B}\bo{u}+\bo{F}\lambda $ into the constraints equation $\bo{G}\dot{\bo{x}}=0$ we get:
		%
		\begin{align}
			\bo{G}(\bo{A}\bo{x}+\bo{B}\bo{u}+\bo{F}\lambda )=0
			\\
			\bo{G}\bo{F}\lambda =-\bo{G}(\bo{A}\bo{x}+\bo{B}\bo{u})
			\\
			\lambda =-(\bo{G}\bo{F})^+\bo{G}(\bo{A}\bo{x}+\bo{B}\bo{u})
			\\
			\dot{\bo{x}}=\bo{A}\bo{x}+\bo{B}\bo{u}-\bo{F}(\bo{G}\bo{F})^+\bo{G}(\bo{A}\bo{x}+\bo{B}
			\\
			\dot{\bo{x}}=(\bo{I}-\bo{F}(\bo{G}\bo{F})^+\bo{G})(\bo{A}\bo{x}+\bo{B}\bo{u})
		\end{align}
		
		Defining $\bo{A}_c = (\bo{I}-\bo{F}(\bo{G}\bo{F})^+\bo{G})\bo{A}$ and $\bo{B}_c = (\bo{I}-\bo{F}(\bo{G}\bo{F})^+\bo{G})\bo{B}$, we obtain LTI system with implicit constraints (IC-LTI):
		%
		\begin{equation}
			\begin{cases}
				\dot{\bo{x}}=\bo{A}_c\bo{x}+\bo{B}_c\bo{u}
				\\
				\bo{G}\dot{\bo{x}}=0
			\end{cases}
		\end{equation}
		
	\end{flushleft}
\end{frame}



\begin{frame}{LTI system with implicit constraints}
	% \framesubtitle{Parameter estimation}
	\begin{flushleft}
		
		\textcolor{mygrey}{
		\begin{equation}
			\begin{cases}
				\dot{\bo{x}}=\bo{A}_c\bo{x}+\bo{B}_c\bo{u}
				\\
				\bo{G}\dot{\bo{x}}=0
			\end{cases}
		\end{equation}
		}

		Note that IC-LTI is similar in form to regular LTI $\dot{\bo{x}}=\bo{A}\bo{x}+\bo{B}\bo{u}$. This reminds us of the use of projectors in manipulator equations. Expression $(\bo{I}-\bo{F}(\bo{G}\bo{F})^+\bo{G})$ acts in a similar ways to projectors.

	\end{flushleft}
\end{frame}



\begin{frame}{Linearization, 1}
	% \framesubtitle{Parameter estimation}
	\begin{flushleft}
		
		We can linearize dynamical system with constraints. Let us remember that linearization is done by Taylor expansion, taking into account the first term. For systems with constraints, we need to apply Taylor expansion to the expression for $\ddot{\bo{q}}$: 
		
		\begin{align}
			\ddot{\bo{q}}
			=
			\bo{f}(\bo{q},  \dot{\bo{q}}, \bo{u})
			=
			(\bo{I}-\bo{J}^\# \bo{J})
			\bo{H}^{-1}(\bo{T}\bo{u} - \bo{C} \dot{\bo{q}} - \bo{g})
			-
			\bo{J}^\# \dot{\bo{J}} \dot{\bo{q}}
		\end{align}
	%
	where $\bo{J}^\# = \bo{H}^{-1} \bo{J}\T ( \bo{J} \bo{H}^{-1} \bo{J}\T )^{-1}$.
		
	\end{flushleft}
\end{frame}



\begin{frame}{Linearization, 2}
	% \framesubtitle{Parameter estimation}
	\begin{flushleft}
		
		Taylor expansion becomes:
		
		\begin{align*}
			\ddot{\bo{q}}
			=
			\bo{f}(\bo{q}_0,  \dot{\bo{q}}_0, \bo{u}_0)
			+
			\frac{\partial \bo{f}}{\partial \bo{q}}
			(\bo{q}-\bo{q}_0)
			+
			\frac{\partial \bo{f}}{\partial \dot{\bo{q}}}
			(\dot{\bo{q}}-\dot{\bo{q}}_0)
			+
			\frac{\partial \bo{f}}{\partial \bo{u}}
			(\bo{u}-\bo{u}_0)
			+
			\text{h.o.t.}
		\end{align*}
	
		Defining $\bo{A}_q = \frac{\partial \bo{f}}{\partial \bo{q}}$ and $\bo{A}_v = \frac{\partial \bo{f}}{\partial \dot{\bo{q}}}$, $\bo{B} = \frac{\partial \bo{f}}{\partial \bo{u}}$, and proposing state variable $\bo{x} = \begin{bmatrix}
			\bo{q} \\ \dot{\bo{q}}
		\end{bmatrix}$
	
	\begin{align}
		\dot{\bo{x}} = 
		\begin{bmatrix}
			0 & \bo{I} \\
			\bo{A}_q & \bo{A}_v
		\end{bmatrix}
	\bo{x}
	+
	\begin{bmatrix}
		0 \\
		\bo{B}
	\end{bmatrix}
	\bo{u}
	+
	\text{const}
	\end{align}
		
	\end{flushleft}
\end{frame}



\begin{frame}{Linearization, 3}
	% \framesubtitle{Parameter estimation}
	\begin{flushleft}
		
		Constraint equation $\bo{J} \ddot{\bo{q}} + \dot{\bo{J}} \dot{\bo{q}}= 0$ becomes:
		%
		\begin{align}
			\begin{bmatrix}
				\bo{J}  & 0 \\
				\dot{\bo{J}} & \bo{J}
			\end{bmatrix}
		\dot{\bo{x}}
		= 0
		\end{align}
		%
		Linearized system then becomes:
		%
			\begin{align}
				\begin{cases}
					\dot{\bo{x}} = 
					\begin{bmatrix}
						0 & \bo{I} \\
						\bo{A}_q & \bo{A}_v
					\end{bmatrix}
					\bo{x}
					+
					\begin{bmatrix}
						0 \\
						\bo{B}
					\end{bmatrix}
					\bo{u}
					+
					\text{const}
					\\
					\begin{bmatrix}
						\bo{J}  & 0 \\
						\dot{\bo{J}} & \bo{J}
					\end{bmatrix}
					\dot{\bo{x}}
					= 0
				\end{cases}
		\end{align}
	
	Noticed that this is an implicit constrained LTI.
		
	\end{flushleft}
\end{frame}




\begin{frame}{Orthogonal decomposition, 1}
	% \framesubtitle{Parameter estimation}
	\begin{flushleft}
		
		We can find orthonormal basis $\bo{N}$ in the null space of the constraints matrix $\bo{G}$, and orthonormal basis $\bo{R}$ in the row space of the same matrix.
		%
			\begin{align}
				\bo{N} = \text{null}(\bo{G}) \\
				\bo{R} = \text{row}(\bo{G}) = \text{col}(\bo{G}\T)
			\end{align}
		
		Since matrices $\bo{N}$ and $\bo{R}$ are orthogonal compliment of each other, the matrix $\begin{bmatrix}
			\bo{N} & \bo{R}
		\end{bmatrix}$ is full rank.
		
		
	\end{flushleft}
\end{frame}




\begin{frame}{Orthogonal decomposition, 2}
	% \framesubtitle{Parameter estimation}
	\begin{flushleft}
		
		We can define new variables $\bo{z}$ and $\zeta$, which have the following relation with $\bo{x}$:
		%
			\begin{align}
				\bo{x} = 
				\begin{bmatrix}
					\bo{N} & \bo{R}
				\end{bmatrix}
				\begin{bmatrix}
					\bo{z} \\ \zeta
				\end{bmatrix}
			\end{align}		
		
		Because $\begin{bmatrix}
			\bo{N} & \bo{R}
		\end{bmatrix}$ is full rank, there is a one-to-one correspondence between variables $\bo{z}$ and $\zeta$ and $\bo{x}$. In fact:
	
		\begin{align}
			\bo{z} = \bo{N}\T \bo{x} \\
			\zeta = \bo{R}\T \bo{x} 
		\end{align}
		
		
	\end{flushleft}
\end{frame}



\begin{frame}{Orthogonal decomposition, 3}
	% \framesubtitle{Parameter estimation}
	\begin{flushleft}
		
		Since $\bo{G} \dot{\bo{x}} = 0$, it follows that $\dot{\bo{x}} \in \text{null}(\bo{G})$. Therefore $\bo{R}\T \dot{\bo{x}} = \dot \zeta =  0$ and $\bo{N}\T \dot{\bo{x}} = \dot{\bo{z}}$:
		%
		\begin{align}
			\dot{\bo{x}} = \begin{bmatrix}
				\bo{N} & \bo{R}
			\end{bmatrix}
			\begin{bmatrix}
		 		\dot{\bo{z}} \\ \dot \zeta
			\end{bmatrix} = \bo{N}\dot{\bo{z}}
		\end{align}
		
		This implies that $\dot{\bo{x}}$ lies in the column space of $\bo{N}$.
		
	\end{flushleft}
\end{frame}



\begin{frame}{Orthogonal projection}
	% \framesubtitle{Parameter estimation}
	\begin{flushleft}
		
		We can multiply equation $\dot{\bo{x}}=\bo{A}_c\bo{x}+\bo{B}_c\bo{u}$ by $\bo{N}\T$. Since $\dot{\bo{x}}$ lies in the column space of $\bo{N}$, we do not lose any component of the dynamics:
		%
		\begin{equation}
			\bo{N}\T\dot{\bo{x}}=\bo{N}\T\bo{A}_c\bo{x}+\bo{N}\T\bo{B}_c\bo{u}
		\end{equation}
	
		Since $\bo{x} = \bo{N} \bo{z} + \bo{R} \zeta$ and $\bo{N}\T \dot{\bo{x}} = \dot{\bo{z}}$ we get:
		%
		\begin{equation}
			\dot{\bo{z}}=\bo{N}\T\bo{A}_c\bo{N} \bo{z} + \bo{N}\T\bo{A}_c\bo{R} \zeta+\bo{N}\T\bo{B}_c\bo{u}
		\end{equation}
		
		We can define $\bo{A}_N = \bo{N}\T\bo{A}_c\bo{N}$, $\bo{A}_R = \bo{N}\T\bo{A}_c\bo{R}$ and $\bo{B}_N = \bo{N}\T\bo{B}_c$:
		%
		\begin{equation}
			\dot{\bo{z}}=\bo{A}_N\bo{z} + \bo{A}_R \zeta+\bo{B}_N\bo{u}
		\end{equation}
		
		
		
	\end{flushleft}
\end{frame}




\begin{frame}{Feedback control}
	% \framesubtitle{Parameter estimation}
	\begin{flushleft}
		
		Given equation $\dot{\bo{z}}=\bo{A}_N\bo{z} + \bo{A}_R \zeta+\bo{B}_N\bo{u}$ we can find stabilizing control. Since $\bo{A}_R \zeta = \text{const}$, this is equivalent to stabilizing the system $\dot{\bo{z}}=\bo{A}_N\bo{z} + \bo{B}_N\bo{u}$.
		
		\bigskip
		
		Consider linear control law:
		
		\begin{equation}
			\bo{u} = -\bo{K}\bo{z}
		\end{equation}
		
		Then dynamics becomes:
		
		\begin{equation}
			\dot{\bo{z}}=(\bo{A}_N - \bo{B}_N\bo{K})\bo{z} + \bo{A}_R \zeta
		\end{equation}
	
		As long at the state matrix $\bo{A}_N - \bo{B}_N\bo{K}$ is Hurwitz, the system is stable. This can be achieved with, e.g. pole placement.
	
	\end{flushleft}
\end{frame}



\begin{frame}{Orthogonal LQR}
	% \framesubtitle{Parameter estimation}
	\begin{flushleft}
		
		With that, we could pose LQR problem for the system. For the dynamical system:
		
		\begin{equation}
			\dot{\bo{z}}=\bo{A}_N\bo{z} +\bo{B}_N\bo{u}
		\end{equation}
		
		find control that minimizes the next cost function:
		
		\begin{equation}
			J = \int \left ( \bo{z}\T \bo{Q} \bo{z} + \bo{u}\T \bo{R} \bo{u} \right ) dt
		\end{equation}
		
		the optimal control is linear control law:
		
		\begin{equation}
			\bo{u}=-\bo{R}^{-1} \bo{B}_N^\top \bo{S} \bo{z}
		\end{equation}
		
		where the control gain $\bo{K} = \bo{R}^{-1} \bo{B}_N^\top \bo{S}$ is obtained by solving algebraic Riccati equation:
		
		\begin{equation}
			\bo{Q} - \bo{S} \bo{B}_N \bo{R}^{-1} \bo{B}_N^\top \bo{S} 
			+ \bo{S} \bo{A}_N + \bo{A}_N^\top \bo{S} = 0
		\end{equation}
		
		
	\end{flushleft}
\end{frame}



\begin{frame}{LQR and Pole placement}
	% \framesubtitle{Parameter estimation}
	\begin{flushleft}
		
		We can consider LQR as a recipe of getting control gain $\bo{K}$:
		
		\begin{equation}
			\bo{K} = \text{lqr}(\bo{A}_N, \bo{B}_N, \bo{Q}, \bo{R})
		\end{equation}
	
		We can do the same with pole placement:
		
		\begin{equation}
			\bo{K} = \text{place}(\bo{A}_N, \bo{B}_N, \bo{p})
		\end{equation}
	
		where $\bo{p}$ are poles of the system.
		
		\bigskip
		
		The main outtake is that we can use standard control design methods from linear control. 
		
	\end{flushleft}
\end{frame}




\begin{frame}{Read more}
	
	\begin{itemize}
		
		\item Mason, Sean, et al. "Balancing and walking using full dynamics LQR control with contact constraints." 2016 IEEE-RAS 16th International Conference on Humanoid Robots (Humanoids). IEEE, 2016. \bref{https://arxiv.org/pdf/1701.08179}{arxiv.org/pdf/1701.08179}
		
		\item Mason, Sean, Ludovic Righetti, and Stefan Schaal. "Full dynamics LQR control of a humanoid robot: An experimental study on balancing and squatting." 2014 IEEE-RAS International Conference on Humanoid Robots. IEEE, 2014. \bref{https://ieeexplore.ieee.org/abstract/document/7041387/}{ieeexplore.ieee.org/abstract/document/7041387/}
		 
		
	\end{itemize}
	
\end{frame}



\myqrframe

\end{document}
