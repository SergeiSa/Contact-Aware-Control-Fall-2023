\documentclass{beamer}

\pdfmapfile{+sansmathaccent.map}


\mode<presentation>
{
	\usetheme{Warsaw} % or try Darmstadt, Madrid, Warsaw, Rochester, CambridgeUS, ...
	\usecolortheme{seahorse} % or try seahorse, beaver, crane, wolverine, ...
	\usefonttheme{serif}  % or try serif, structurebold, ...
	\setbeamertemplate{navigation symbols}{}
	\setbeamertemplate{caption}[numbered]
} 


%%%%%%%%%%%%%%%%%%%%%%%%%%%%
% itemize settings


%%%%%%%%%%%%%%%%%%%%%%%%%%%%
% itemize settings

\definecolor{myhotpink}{RGB}{255, 80, 200}
\definecolor{mywarmpink}{RGB}{255, 60, 160}
\definecolor{mylightpink}{RGB}{255, 80, 200}
\definecolor{mypink}{RGB}{255, 30, 80}
\definecolor{mydarkpink}{RGB}{155, 25, 60}

\definecolor{mypaleblue}{RGB}{240, 240, 255}
\definecolor{mylightblue}{RGB}{120, 150, 255}
\definecolor{myblue}{RGB}{90, 90, 255}
\definecolor{mygblue}{RGB}{70, 110, 240}
\definecolor{mydarkblue}{RGB}{0, 0, 180}
\definecolor{myblackblue}{RGB}{40, 40, 120}

\definecolor{myblackturquoise}{RGB}{5, 53, 60}
\definecolor{mydarkdarkturquoise}{RGB}{8, 93, 110}
\definecolor{mydarkturquoise}{RGB}{28, 143, 150}
\definecolor{mypaleturquoise}{RGB}{230, 255, 255}
\definecolor{myturquoise}{RGB}{48, 213, 200}

\definecolor{mygreen}{RGB}{0, 200, 0}
\definecolor{mydarkgreen}{RGB}{0, 120, 0}
\definecolor{mygreen2}{RGB}{245, 255, 230}

\definecolor{mygrey}{RGB}{120, 120, 120}
\definecolor{mypalegrey}{RGB}{160, 160, 160}
\definecolor{mydarkgrey}{RGB}{80, 80, 160}

\definecolor{mydarkred}{RGB}{160, 30, 30}
\definecolor{mylightred}{RGB}{255, 150, 150}
\definecolor{myred}{RGB}{200, 110, 110}
\definecolor{myblackred}{RGB}{120, 40, 40}


\definecolor{myblackmaroon}{RGB}{50, 0, 15}

\definecolor{mygreen}{RGB}{0, 200, 0}
\definecolor{mygreen2}{RGB}{205, 255, 200}

\definecolor{mydarkcolor}{RGB}{60, 25, 155}
\definecolor{mylightcolor}{RGB}{130, 180, 250}

\setbeamertemplate{itemize items}[default]

\setbeamertemplate{itemize item}{\color{myblackmaroon}$\blacksquare$}
\setbeamertemplate{itemize subitem}{\color{mydarkdarkturquoise}$\blacktriangleright$}
\setbeamertemplate{itemize subsubitem}{\color{mygray}$\blacksquare$}

\setbeamercolor{palette quaternary}{fg=white,bg=myblackmaroon}
\setbeamercolor{titlelike}{parent=palette quaternary}

\setbeamercolor{palette quaternary2}{fg=black,bg=mypaleblue}
\setbeamercolor{frametitle}{parent=palette quaternary2}

\setbeamerfont{frametitle}{size=\Large,series=\scshape}
\setbeamerfont{framesubtitle}{size=\normalsize,series=\upshape}





%%%%%%%%%%%%%%%%%%%%%%%%%%%%
% block settings

\setbeamercolor{block title}{bg=red!30,fg=black}

\setbeamercolor*{block title example}{bg=mygreen!40!white,fg=black}

\setbeamercolor*{block body example}{fg= black, bg= mygreen2}


%%%%%%%%%%%%%%%%%%%%%%%%%%%%
% URL settings
\hypersetup{
	colorlinks=true,
	linkcolor=blue,
	filecolor=blue,      
	urlcolor=blue,
}

%%%%%%%%%%%%%%%%%%%%%%%%%%

\renewcommand{\familydefault}{\rmdefault}

\usepackage{amsmath}
\usepackage{mathtools}

\usepackage{subcaption}

\usepackage{qrcode}

\DeclareMathOperator*{\argmin}{arg\,min}
\newcommand{\bo}[1] {\mathbf{#1}}

\newcommand{\R}{\mathbb{R}} 
\newcommand{\T}{^\top}     



\newcommand{\mydate}{Fall 2023}

\newcommand{\mygit}{\textcolor{blue}{\href{https://github.com/SergeiSa/Mechatronics-2023}{github.com/SergeiSa/Mechatronics-2023}}}

\newcommand{\myqr}{ \textcolor{black}{\qrcode[height=1.5in]{https://github.com/SergeiSa/Mechatronics-2023}}
}

\newcommand{\myqrframe}{
	\begin{frame}
		\centerline{Lecture slides are available via Github, links are on Moodle}
		\bigskip
		\centerline{You can help improve these slides at:}
		\centerline{\mygit}
		\bigskip
		\myqr
	\end{frame}
}


\newcommand{\bref}[2] {\textcolor{blue}{\href{#1}{#2}}}

%%%%%%%%%%%%%%%%%%%%%%%%%%%%
% code settings

\usepackage{listings}
\usepackage{color}
% \definecolor{mygreen}{rgb}{0,0.6,0}
% \definecolor{mygray}{rgb}{0.5,0.5,0.5}
\definecolor{mymauve}{rgb}{0.58,0,0.82}
\lstset{ 
	backgroundcolor=\color{white},   % choose the background color; you must add \usepackage{color} or \usepackage{xcolor}; should come as last argument
	basicstyle=\footnotesize,        % the size of the fonts that are used for the code
	breakatwhitespace=false,         % sets if automatic breaks should only happen at whitespace
	breaklines=true,                 % sets automatic line breaking
	captionpos=b,                    % sets the caption-position to bottom
	commentstyle=\color{mygreen},    % comment style
	deletekeywords={...},            % if you want to delete keywords from the given language
	escapeinside={\%*}{*)},          % if you want to add LaTeX within your code
	extendedchars=true,              % lets you use non-ASCII characters; for 8-bits encodings only, does not work with UTF-8
	firstnumber=0000,                % start line enumeration with line 0000
	frame=single,	                   % adds a frame around the code
	keepspaces=true,                 % keeps spaces in text, useful for keeping indentation of code (possibly needs columns=flexible)
	keywordstyle=\color{blue},       % keyword style
	language=Octave,                 % the language of the code
	morekeywords={*,...},            % if you want to add more keywords to the set
	numbers=left,                    % where to put the line-numbers; possible values are (none, left, right)
	numbersep=5pt,                   % how far the line-numbers are from the code
	numberstyle=\tiny\color{mygray}, % the style that is used for the line-numbers
	rulecolor=\color{black},         % if not set, the frame-color may be changed on line-breaks within not-black text (e.g. comments (green here))
	showspaces=false,                % show spaces everywhere adding particular underscores; it overrides 'showstringspaces'
	showstringspaces=false,          % underline spaces within strings only
	showtabs=false,                  % show tabs within strings adding particular underscores
	stepnumber=2,                    % the step between two line-numbers. If it's 1, each line will be numbered
	stringstyle=\color{mymauve},     % string literal style
	tabsize=2,	                   % sets default tabsize to 2 spaces
	title=\lstname                   % show the filename of files included with \lstinputlisting; also try caption instead of title
}


%%%%%%%%%%%%%%%%%%%%%%%%%%%%
% URL settings
\hypersetup{
	colorlinks=false,
	linkcolor=blue,
	filecolor=blue,      
	urlcolor=blue,
}

%%%%%%%%%%%%%%%%%%%%%%%%%%

%%%%%%%%%%%%%%%%%%%%%%%%%%%%
% tikz settings

\usepackage{tikz}
\tikzset{every picture/.style={line width=0.75pt}}


\title{Orthogonal Observer}
\subtitle{Contact-aware Control, Lecture 6}
\author{by Sergei Savin}
\centering
\date{\mydate}



\begin{document}
\maketitle


\begin{frame}{Content}

\begin{itemize}
\item recap
\item Observation
\item Subspace representation
\item Orthogonal Observer
\item Separation Principle
\end{itemize}

\end{frame}




\begin{frame}{recap - LTI system with constraints}
	% \framesubtitle{Parameter estimation}
	\begin{flushleft}
		
		LTI system with explicit constraints (EC-LTI) can be presented in the following form:
		%
		\begin{equation}
			\begin{cases}
				\dot{\bo{x}}=\bo{A}\bo{x}+\bo{B}\bo{u}+\bo{F}\lambda 
				\\
				\bo{G}\dot{\bo{x}}=0
			\end{cases}
		\end{equation}
		
		It is equivalent to LTI system with implicit constraints (IC-LTI):
		%
		\begin{equation}
			\dot{\bo{x}}=\bo{A}_c\bo{x}+\bo{B}_c\bo{u}
		\end{equation}
	%
	where $\bo{A}_c = (\bo{I}-\bo{F}(\bo{G}\bo{F})^+\bo{G})\bo{A}$ and $\bo{B}_c = (\bo{I}-\bo{F}(\bo{G}\bo{F})^+\bo{G})\bo{B}$.
	
	\end{flushleft}
\end{frame}


\begin{frame}{recap - Orthogonal LQR}
	% \framesubtitle{Parameter estimation}
	\begin{flushleft}
		
		We define $\bo{N} = \text{null}(\bo{G})$ and $\bo{R} = \text{col}(\bo{G}\T)$, giving us $\bo{x} = \bo{N}\bo{z}+\bo{R}\zeta$ and $\dot{\bo{x}} = \bo{N}\dot{\bo{z}}$. Multiplying the IC-LTI by $\bo{N}\T$ we get:
		
		\begin{equation}
			\dot{\bo{z}}=\bo{N}\T\bo{A}_c(\bo{N}\bo{z}+\bo{R}\zeta)+\bo{N}\T\bo{B}_c\bo{u}
		\end{equation}
		
		Orthogonal LQR is solved just as the regular one, but with different quadruple $(\textcolor{myred}{\bo{A}}, \textcolor{myblue}{\bo{B}}, \bo{Q}, \bo{R})$:
		
		\begin{equation}
			\bo{K}_z = \text{lqr}(\textcolor{myred}{(\bo{N}\T\bo{A}_c\bo{N})}, \ \textcolor{myblue}{(\bo{N}\T\bo{B}_c)}, \ \bo{Q}_N, \ \bo{R})
		\end{equation}
		%
		which minimizes the cost function:
	
		\begin{equation}
			J = \int \left ( \bo{z}\T \bo{Q}_N \bo{z} + \bo{u}\T \bo{R} \bo{u} \right ) dt
		\end{equation}
		
		
	\end{flushleft}
\end{frame}



\begin{frame}{Observation}
	% \framesubtitle{Parameter estimation}
	\begin{flushleft}
		
		We can define IC-LTI with observation output $\bo{y}$:
		%
		\begin{equation}
			\begin{cases}
				\dot{\bo{x}}=\bo{A}_c\bo{x}+\bo{B}_c\bo{u}
				\\
				\bo{y} = \bo{C}\bo{x}
			\end{cases}
	\end{equation}

	It can be observable or not; however, we can take advantage of the constraints to lower the number of states we need to observe.
	
	\bigskip
	
	As before, we multiply the dynamics by $\bo{N}\T$ and replace $\bo{x}$ by pair $(\bo{z}, \zeta)$:
	%
	\begin{equation}
		\begin{cases}
			\dot{\bo{z}}=\bo{N}\T\bo{A}_c\bo{N}\bo{z} + \bo{N}\T\bo{A}_c\bo{R}\zeta + \bo{N}\T\bo{B}_c\bo{u}
			\\
			\bo{y} = \bo{C}\bo{N}\bo{z} + \bo{C}\bo{R}\zeta
		\end{cases}
	\end{equation}
		
	\end{flushleft}
\end{frame}




\begin{frame}{Subspace representation}
	% \framesubtitle{Parameter estimation}
	\begin{flushleft}
		
		We can re-write the dynamics in terms of the pair $(\bo{z}, \zeta)$:
		%
		\begin{equation}
			\label{e_CLTI_z_zeta}
			\begin{cases}
				\begin{bmatrix}
					\dot{\bo{z}} \\ \dot{\zeta}
				\end{bmatrix} =
				\begin{bmatrix}
					\bo{N}\T\bo{A}_c\bo{N} & \bo{N}\T\bo{A}_c\bo{R}\\ 
					\bo{0} & \bo{0}
				\end{bmatrix}
				\begin{bmatrix}
					\bo{z} \\ \zeta
				\end{bmatrix}
				+
				\begin{bmatrix}
					\bo{N}\T\bo{B}_c\bo{u} \\ \bo{0}
				\end{bmatrix} \\
				\bo{y} = \bo{C} 
				\begin{bmatrix}
					\bo{N} & \bo{R}
				\end{bmatrix}
				\begin{bmatrix}
					\bo{z} \\ \zeta
				\end{bmatrix}
			\end{cases}
		\end{equation}
		
		We will call it \emph{subspace representation}.
		
	\end{flushleft}
\end{frame}



\begin{frame}{Orthogonal Observer}
	% \framesubtitle{Parameter estimation}
	\begin{flushleft}
		
		Using this representation we can propose the following state observer:
		
		\begin{equation*}
			\begin{bmatrix}
				\hat{\dot{\bo{z}}} \\ \dot{\hat{\zeta}}
			\end{bmatrix} =
			\begin{bmatrix}
				\bo{N}\T\bo{A}_c\bo{N} & \bo{N}\T\bo{A}_c\bo{R}\\ 
				\bo{0} & \bo{0}
			\end{bmatrix}
			\begin{bmatrix}
				\hat{\bo{z}} \\ \hat{\zeta}
			\end{bmatrix}
			+
			\begin{bmatrix}
				\bo{N}\T\bo{B}_c\bo{u} \\ \bo{0}
			\end{bmatrix} 
			+
			\bo{L}
			\left(\bo{y} - \bo{C} 
			\begin{bmatrix}
				\bo{N} & \bo{R}
			\end{bmatrix}
			\begin{bmatrix}
				\hat{\bo{z}} \\ \hat{\zeta}
			\end{bmatrix}
			\right),
		\end{equation*}
		%
		where $\hat{\bo{z}}$ and $\hat{\zeta}$ are estimates of $\bo{z}$ and $\zeta$.
		
	\end{flushleft}
\end{frame}



\begin{frame}{Orthogonal Observer - gain design}
	% \framesubtitle{Parameter estimation}
	\begin{flushleft}
		
		\begin{Theorem}
			\label{SeparationPrinciple}
			Orthogonal Observer and IC-LTI dynamics, with control law $\bo{u} = -\bo{K}_z \hat{\bo{z}} - \bo{K}_\zeta \hat{\zeta}$ form a stable system as the following condition: are satisfied:
			%
			\begin{equation}
				\label{e_Observer_stability}
				\begin{bmatrix}
					\bo{N}\T \\ \bo{R}\T
				\end{bmatrix}
				\left(
				\begin{bmatrix}\bo{A}_c\T\bo{N} \\ \bo{0}
				\end{bmatrix}
				- 
				\bo{C}\T \bo{L}\T
				\right)
				\in \mathbb{H}
			\end{equation}
		\end{Theorem}
	
		Here we assume that $\bo{K}_z$ and $\bo{K}_\zeta$ are chosen such that:
		%
		\begin{align}
			\bo{N}\T\bo{A}_c\bo{N} - \bo{N}\T\bo{B}_c\bo{K}_z \in \mathbb{H} \\
			\bo{K}_\zeta = (\bo{N}\T\bo{B}_c)^+\bo{N}\T\bo{A}_c\bo{R}
		\end{align}
		
		
	\end{flushleft}
\end{frame}



\begin{frame}{Separation Principle, 1}
	% \framesubtitle{Parameter estimation}
	\begin{flushleft}
		
		Defining observation error 
		$\bo{e} = 
		\begin{bmatrix}
			\bo{z} - \hat{\bo{z}} \\ \zeta - \hat{\zeta}
		\end{bmatrix}$
		and subtracting observer from system dynamics we get observer error dynamics:
		%
		\begin{equation*}
			\begin{cases}
				\begin{bmatrix}
					\dot{\bo{z}} \\ \dot{\zeta}
				\end{bmatrix} =
				\textcolor{myred}
				{\begin{bmatrix}
					\bo{N}\T\bo{A}_c\bo{N} & \bo{N}\T\bo{A}_c\bo{R}\\ 
					\bo{0} & \bo{0}
				\end{bmatrix}}
				\begin{bmatrix}
					\bo{z} \\ \zeta
				\end{bmatrix}
				+
				\textcolor{myblue}
				{\begin{bmatrix}
					\bo{N}\T\bo{B}_c\bo{u} \\ \bo{0}
				\end{bmatrix}} \\
				\bo{y} = 
				\textcolor{mydarkgreen}
				{\bo{C} 
				\begin{bmatrix}
					\bo{N} & \bo{R}
				\end{bmatrix}}
				\begin{bmatrix}
					\bo{z} \\ \zeta
				\end{bmatrix}
			\end{cases}
		\end{equation*}
		%
		\begin{equation*}
			\begin{bmatrix}
				\hat{\dot{\bo{z}}} \\ \dot{\hat{\zeta}}
			\end{bmatrix} =
		\textcolor{myred}
			{\begin{bmatrix}
				\bo{N}\T\bo{A}_c\bo{N} & \bo{N}\T\bo{A}_c\bo{R}\\ 
				\bo{0} & \bo{0}
			\end{bmatrix}}
			\begin{bmatrix}
				\hat{\bo{z}} \\ \hat{\zeta}
			\end{bmatrix}
			+
			\textcolor{myblue}
			{\begin{bmatrix}
				\bo{N}\T\bo{B}_c\bo{u} \\ \bo{0}
			\end{bmatrix} }
			+
			\bo{L}
			\left(\bo{y} - 
			\textcolor{mydarkgreen}
			{\bo{C} 
			\begin{bmatrix}
				\bo{N} & \bo{R}
			\end{bmatrix}}
			\begin{bmatrix}
				\hat{\bo{z}} \\ \hat{\zeta}
			\end{bmatrix}
			\right),
		\end{equation*}
		%
		\begin{equation}
			\label{e_Observer_error}
			\dot{\bo{e}} =
			\left(
			\textcolor{myred}
			{\begin{bmatrix}
				\bo{N}\T\bo{A}_c\bo{N} & \bo{N}\T\bo{A}_c\bo{R}\\ 
				\bo{0} & \bo{0}
			\end{bmatrix}}
			-
			\bo{L}
			\textcolor{mydarkgreen}
			{\bo{C} 
			\begin{bmatrix}
				\bo{N} & \bo{R}
			\end{bmatrix}}
			\right)
			\bo{e}
		\end{equation}
		
		
	\end{flushleft}
\end{frame}



\begin{frame}{Separation Principle, 2}
	% \framesubtitle{Parameter estimation}
	\begin{flushleft}
		
		Substituting control law, we find state dynamics as:
		
		\begin{align*}
			\dot{\bo{z}} &= 
			(\bo{N}\T\bo{A}_c\bo{N} - \bo{N}\T\bo{B}_c 
			\bo{K}_z) \bo{z}
			+
			\\
			&+
			\bo{N}\T\bo{B}_c\bo{K}
			\bo{e}
			+ (\bo{N}\T\bo{A}_c\bo{R}
			- \bo{N}\T\bo{B}_c \bo{K}_\zeta) \zeta.
		\end{align*}
		%
		where $\bo{K} = \begin{bmatrix}
			\bo{K}_z & \bo{K}_\zeta
		\end{bmatrix}$.
		
		
	\end{flushleft}
\end{frame}



\begin{frame}{Separation Principle, 3}
	% \framesubtitle{Parameter estimation}
	\begin{flushleft}
		
		
		With that we can write the combined state and observer dynamics:
		
		\begin{equation}
			\label{e_ObserverError_State}
			\begin{bmatrix}
				\dot{\bo{z}} \\ \dot{\bo{e}}
			\end{bmatrix} 
			= 
			\begin{bmatrix}
				(\bo{N}\T\bo{A}_c\bo{N} - \bo{N}\T\bo{B}_c 
				\bo{K}_z) & \bo{N}\T\bo{B}_c\bo{K} \\
				\bo{0} & (\bar{\bo{N}}\T\bo{A}_c - \bo{L}\bo{C})\bo{E}
			\end{bmatrix}
			\begin{bmatrix}
				\bo{z} \\ \bo{e}
			\end{bmatrix} 
			+
			\text{const}
		\end{equation}
		%
		where 
		$\bar{\bo{N}} = \begin{bmatrix}
			\bo{N} & \bo{0}_{n \times n}
		\end{bmatrix}$, 
		and
		$\bo{E} = \begin{bmatrix}
			\bo{N} & \bo{R}
		\end{bmatrix}$. 
	%
		Since the state matrix here is upper triangular, we only need the diagonal blocks $(\bo{N}\T\bo{A}_c\bo{N} - \bo{N}\T\bo{B}_c\bo{K}_z)$ and $(\bar{\bo{N}}\T\bo{A}_c - \bo{L}\bo{C})\bo{E}$ to be stable for the system to be stable. Transpose of the last one gives us proof of the theorem. $\qed$
		
		\bigskip
		
		Note that we derived a \emph{separation principle} for the system with constraints.
		
	\end{flushleft}
\end{frame}



\begin{frame}{Read more}
	
	\begin{itemize}
		
		\item \bref{https://www.mdpi.com/1424-8220/21/18/6312}{Savin, S., Balakhnov, O., Khusainov, R. and Klimchik, A., 2021. State observer for linear systems with explicit constraints: Orthogonal decomposition method. Sensors, 21(18), p.6312}
		
	\end{itemize}
	
\end{frame}



\myqrframe

\end{document}
