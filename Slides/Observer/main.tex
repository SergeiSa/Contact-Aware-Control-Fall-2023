\documentclass{beamer}

\input{settings.tex}


\title{Orthogonal Observer}
\subtitle{Contact-aware Control, Lecture 6}
\author{by Sergei Savin}
\centering
\date{\mydate}



\begin{document}
\maketitle


\begin{frame}{Content}

\begin{itemize}
\item recap
\item Observation
\item Subspace representation
\item Orthogonal Observer
\item Separation Principle
\end{itemize}

\end{frame}




\begin{frame}{recap - LTI system with constraints}
	% \framesubtitle{Parameter estimation}
	\begin{flushleft}
		
		LTI system with explicit constraints (EC-LTI) can be presented in the following form:
		%
		\begin{equation}
			\begin{cases}
				\dot{\bo{x}}=\bo{A}\bo{x}+\bo{B}\bo{u}+\bo{F}\lambda 
				\\
				\bo{G}\dot{\bo{x}}=0
			\end{cases}
		\end{equation}
		
		It is equivalent to LTI system with implicit constraints (IC-LTI):
		%
		\begin{equation}
			\dot{\bo{x}}=\bo{A}_c\bo{x}+\bo{B}_c\bo{u}
		\end{equation}
	%
	where $\bo{A}_c = (\bo{I}-\bo{F}(\bo{G}\bo{F})^+\bo{G})\bo{A}$ and $\bo{B}_c = (\bo{I}-\bo{F}(\bo{G}\bo{F})^+\bo{G})\bo{B}$.
	
	\end{flushleft}
\end{frame}


\begin{frame}{recap - Orthogonal LQR}
	% \framesubtitle{Parameter estimation}
	\begin{flushleft}
		
		We define $\bo{N} = \text{null}(\bo{G})$ and $\bo{R} = \text{col}(\bo{G}\T)$, giving us $\bo{x} = \bo{N}\bo{z}+\bo{R}\zeta$ and $\dot{\bo{x}} = \bo{N}\dot{\bo{z}}$. Multiplying the IC-LTI by $\bo{N}\T$ we get:
		
		\begin{equation}
			\dot{\bo{z}}=\bo{N}\T\bo{A}_c(\bo{N}\bo{z}+\bo{R}\zeta)+\bo{N}\T\bo{B}_c\bo{u}
		\end{equation}
		
		Orthogonal LQR is solved just as the regular one, but with different quadruple $(\textcolor{myred}{\bo{A}}, \textcolor{myblue}{\bo{B}}, \bo{Q}, \bo{R})$:
		
		\begin{equation}
			\bo{K}_z = \text{lqr}(\textcolor{myred}{(\bo{N}\T\bo{A}_c\bo{N})}, \ \textcolor{myblue}{(\bo{N}\T\bo{B}_c)}, \ \bo{Q}_N, \ \bo{R})
		\end{equation}
		%
		which minimizes the cost function:
	
		\begin{equation}
			J = \int \left ( \bo{z}\T \bo{Q}_N \bo{z} + \bo{u}\T \bo{R} \bo{u} \right ) dt
		\end{equation}
		
		
	\end{flushleft}
\end{frame}



\begin{frame}{Observation}
	% \framesubtitle{Parameter estimation}
	\begin{flushleft}
		
		We can define IC-LTI with observation output $\bo{y}$:
		%
		\begin{equation}
			\begin{cases}
				\dot{\bo{x}}=\bo{A}_c\bo{x}+\bo{B}_c\bo{u}
				\\
				\bo{y} = \bo{C}\bo{x}
			\end{cases}
	\end{equation}

	It can be observable or not; however, we can take advantage of the constraints to lower the number of states we need to observe.
	
	\bigskip
	
	As before, we multiply the dynamics by $\bo{N}\T$ and replace $\bo{x}$ by pair $(\bo{z}, \zeta)$:
	%
	\begin{equation}
		\begin{cases}
			\dot{\bo{z}}=\bo{N}\T\bo{A}_c\bo{N}\bo{z} + \bo{N}\T\bo{A}_c\bo{R}\zeta + \bo{N}\T\bo{B}_c\bo{u}
			\\
			\bo{y} = \bo{C}\bo{N}\bo{z} + \bo{C}\bo{R}\zeta
		\end{cases}
	\end{equation}
		
	\end{flushleft}
\end{frame}




\begin{frame}{Subspace representation}
	% \framesubtitle{Parameter estimation}
	\begin{flushleft}
		
		We can re-write the dynamics in terms of the pair $(\bo{z}, \zeta)$:
		%
		\begin{equation}
			\label{e_CLTI_z_zeta}
			\begin{cases}
				\begin{bmatrix}
					\dot{\bo{z}} \\ \dot{\zeta}
				\end{bmatrix} =
				\begin{bmatrix}
					\bo{N}\T\bo{A}_c\bo{N} & \bo{N}\T\bo{A}_c\bo{R}\\ 
					\bo{0} & \bo{0}
				\end{bmatrix}
				\begin{bmatrix}
					\bo{z} \\ \zeta
				\end{bmatrix}
				+
				\begin{bmatrix}
					\bo{N}\T\bo{B}_c\bo{u} \\ \bo{0}
				\end{bmatrix} \\
				\bo{y} = \bo{C} 
				\begin{bmatrix}
					\bo{N} & \bo{R}
				\end{bmatrix}
				\begin{bmatrix}
					\bo{z} \\ \zeta
				\end{bmatrix}
			\end{cases}
		\end{equation}
		
		We will call it \emph{subspace representation}.
		
	\end{flushleft}
\end{frame}



\begin{frame}{Orthogonal Observer}
	% \framesubtitle{Parameter estimation}
	\begin{flushleft}
		
		Using this representation we can propose the following state observer:
		
		\begin{equation*}
			\begin{bmatrix}
				\hat{\dot{\bo{z}}} \\ \dot{\hat{\zeta}}
			\end{bmatrix} =
			\begin{bmatrix}
				\bo{N}\T\bo{A}_c\bo{N} & \bo{N}\T\bo{A}_c\bo{R}\\ 
				\bo{0} & \bo{0}
			\end{bmatrix}
			\begin{bmatrix}
				\hat{\bo{z}} \\ \hat{\zeta}
			\end{bmatrix}
			+
			\begin{bmatrix}
				\bo{N}\T\bo{B}_c\bo{u} \\ \bo{0}
			\end{bmatrix} 
			+
			\bo{L}
			\left(\bo{y} - \bo{C} 
			\begin{bmatrix}
				\bo{N} & \bo{R}
			\end{bmatrix}
			\begin{bmatrix}
				\hat{\bo{z}} \\ \hat{\zeta}
			\end{bmatrix}
			\right),
		\end{equation*}
		%
		where $\hat{\bo{z}}$ and $\hat{\zeta}$ are estimates of $\bo{z}$ and $\zeta$.
		
	\end{flushleft}
\end{frame}



\begin{frame}{Orthogonal Observer - gain design}
	% \framesubtitle{Parameter estimation}
	\begin{flushleft}
		
		\begin{Theorem}
			\label{SeparationPrinciple}
			Orthogonal Observer and IC-LTI dynamics, with control law $\bo{u} = -\bo{K}_z \hat{\bo{z}} - \bo{K}_\zeta \hat{\zeta}$ form a stable system as the following condition: are satisfied:
			%
			\begin{equation}
				\label{e_Observer_stability}
				\begin{bmatrix}
					\bo{N}\T \\ \bo{R}\T
				\end{bmatrix}
				\left(
				\begin{bmatrix}\bo{A}_c\T\bo{N} \\ \bo{0}
				\end{bmatrix}
				- 
				\bo{C}\T \bo{L}\T
				\right)
				\in \mathbb{H}
			\end{equation}
		\end{Theorem}
	
		Here we assume that $\bo{K}_z$ and $\bo{K}_\zeta$ are chosen such that:
		%
		\begin{align}
			\bo{N}\T\bo{A}_c\bo{N} - \bo{N}\T\bo{B}_c\bo{K}_z \in \mathbb{H} \\
			\bo{K}_\zeta = (\bo{N}\T\bo{B}_c)^+\bo{N}\T\bo{A}_c\bo{R}
		\end{align}
		
		
	\end{flushleft}
\end{frame}



\begin{frame}{Separation Principle, 1}
	% \framesubtitle{Parameter estimation}
	\begin{flushleft}
		
		Defining observation error 
		$\bo{e} = 
		\begin{bmatrix}
			\bo{z} - \hat{\bo{z}} \\ \zeta - \hat{\zeta}
		\end{bmatrix}$
		and subtracting observer from system dynamics we get observer error dynamics:
		%
		\begin{equation*}
			\begin{cases}
				\begin{bmatrix}
					\dot{\bo{z}} \\ \dot{\zeta}
				\end{bmatrix} =
				\textcolor{myred}
				{\begin{bmatrix}
					\bo{N}\T\bo{A}_c\bo{N} & \bo{N}\T\bo{A}_c\bo{R}\\ 
					\bo{0} & \bo{0}
				\end{bmatrix}}
				\begin{bmatrix}
					\bo{z} \\ \zeta
				\end{bmatrix}
				+
				\textcolor{myblue}
				{\begin{bmatrix}
					\bo{N}\T\bo{B}_c\bo{u} \\ \bo{0}
				\end{bmatrix}} \\
				\bo{y} = 
				\textcolor{mydarkgreen}
				{\bo{C} 
				\begin{bmatrix}
					\bo{N} & \bo{R}
				\end{bmatrix}}
				\begin{bmatrix}
					\bo{z} \\ \zeta
				\end{bmatrix}
			\end{cases}
		\end{equation*}
		%
		\begin{equation*}
			\begin{bmatrix}
				\hat{\dot{\bo{z}}} \\ \dot{\hat{\zeta}}
			\end{bmatrix} =
		\textcolor{myred}
			{\begin{bmatrix}
				\bo{N}\T\bo{A}_c\bo{N} & \bo{N}\T\bo{A}_c\bo{R}\\ 
				\bo{0} & \bo{0}
			\end{bmatrix}}
			\begin{bmatrix}
				\hat{\bo{z}} \\ \hat{\zeta}
			\end{bmatrix}
			+
			\textcolor{myblue}
			{\begin{bmatrix}
				\bo{N}\T\bo{B}_c\bo{u} \\ \bo{0}
			\end{bmatrix} }
			+
			\bo{L}
			\left(\bo{y} - 
			\textcolor{mydarkgreen}
			{\bo{C} 
			\begin{bmatrix}
				\bo{N} & \bo{R}
			\end{bmatrix}}
			\begin{bmatrix}
				\hat{\bo{z}} \\ \hat{\zeta}
			\end{bmatrix}
			\right),
		\end{equation*}
		%
		\begin{equation}
			\label{e_Observer_error}
			\dot{\bo{e}} =
			\left(
			\textcolor{myred}
			{\begin{bmatrix}
				\bo{N}\T\bo{A}_c\bo{N} & \bo{N}\T\bo{A}_c\bo{R}\\ 
				\bo{0} & \bo{0}
			\end{bmatrix}}
			-
			\bo{L}
			\textcolor{mydarkgreen}
			{\bo{C} 
			\begin{bmatrix}
				\bo{N} & \bo{R}
			\end{bmatrix}}
			\right)
			\bo{e}
		\end{equation}
		
		
	\end{flushleft}
\end{frame}



\begin{frame}{Separation Principle, 2}
	% \framesubtitle{Parameter estimation}
	\begin{flushleft}
		
		Substituting control law, we find state dynamics as:
		
		\begin{align*}
			\dot{\bo{z}} &= 
			(\bo{N}\T\bo{A}_c\bo{N} - \bo{N}\T\bo{B}_c 
			\bo{K}_z) \bo{z}
			+
			\\
			&+
			\bo{N}\T\bo{B}_c\bo{K}
			\bo{e}
			+ (\bo{N}\T\bo{A}_c\bo{R}
			- \bo{N}\T\bo{B}_c \bo{K}_\zeta) \zeta.
		\end{align*}
		%
		where $\bo{K} = \begin{bmatrix}
			\bo{K}_z & \bo{K}_\zeta
		\end{bmatrix}$.
		
		
	\end{flushleft}
\end{frame}



\begin{frame}{Separation Principle, 3}
	% \framesubtitle{Parameter estimation}
	\begin{flushleft}
		
		
		With that we can write the combined state and observer dynamics:
		
		\begin{equation}
			\label{e_ObserverError_State}
			\begin{bmatrix}
				\dot{\bo{z}} \\ \dot{\bo{e}}
			\end{bmatrix} 
			= 
			\begin{bmatrix}
				(\bo{N}\T\bo{A}_c\bo{N} - \bo{N}\T\bo{B}_c 
				\bo{K}_z) & \bo{N}\T\bo{B}_c\bo{K} \\
				\bo{0} & (\bar{\bo{N}}\T\bo{A}_c - \bo{L}\bo{C})\bo{E}
			\end{bmatrix}
			\begin{bmatrix}
				\bo{z} \\ \bo{e}
			\end{bmatrix} 
			+
			\text{const}
		\end{equation}
		%
		where 
		$\bar{\bo{N}} = \begin{bmatrix}
			\bo{N} & \bo{0}_{n \times n}
		\end{bmatrix}$, 
		and
		$\bo{E} = \begin{bmatrix}
			\bo{N} & \bo{R}
		\end{bmatrix}$. 
	%
		Since the state matrix here is upper triangular, we only need the diagonal blocks $(\bo{N}\T\bo{A}_c\bo{N} - \bo{N}\T\bo{B}_c\bo{K}_z)$ and $(\bar{\bo{N}}\T\bo{A}_c - \bo{L}\bo{C})\bo{E}$ to be stable for the system to be stable. Transpose of the last one gives us proof of the theorem. $\qed$
		
		\bigskip
		
		Note that we derived a \emph{separation principle} for the system with constraints.
		
	\end{flushleft}
\end{frame}



\begin{frame}{Read more}
	
	\begin{itemize}
		
		\item \bref{https://www.mdpi.com/1424-8220/21/18/6312}{Savin, S., Balakhnov, O., Khusainov, R. and Klimchik, A., 2021. State observer for linear systems with explicit constraints: Orthogonal decomposition method. Sensors, 21(18), p.6312}
		
	\end{itemize}
	
\end{frame}



\myqrframe

\end{document}
