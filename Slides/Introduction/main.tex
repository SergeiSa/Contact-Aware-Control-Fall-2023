\documentclass{beamer}

\pdfmapfile{+sansmathaccent.map}


\mode<presentation>
{
	\usetheme{Warsaw} % or try Darmstadt, Madrid, Warsaw, Rochester, CambridgeUS, ...
	\usecolortheme{seahorse} % or try seahorse, beaver, crane, wolverine, ...
	\usefonttheme{serif}  % or try serif, structurebold, ...
	\setbeamertemplate{navigation symbols}{}
	\setbeamertemplate{caption}[numbered]
} 


%%%%%%%%%%%%%%%%%%%%%%%%%%%%
% itemize settings


%%%%%%%%%%%%%%%%%%%%%%%%%%%%
% itemize settings

\definecolor{myhotpink}{RGB}{255, 80, 200}
\definecolor{mywarmpink}{RGB}{255, 60, 160}
\definecolor{mylightpink}{RGB}{255, 80, 200}
\definecolor{mypink}{RGB}{255, 30, 80}
\definecolor{mydarkpink}{RGB}{155, 25, 60}

\definecolor{mypaleblue}{RGB}{240, 240, 255}
\definecolor{mylightblue}{RGB}{120, 150, 255}
\definecolor{myblue}{RGB}{90, 90, 255}
\definecolor{mygblue}{RGB}{70, 110, 240}
\definecolor{mydarkblue}{RGB}{0, 0, 180}
\definecolor{myblackblue}{RGB}{40, 40, 120}

\definecolor{myblackturquoise}{RGB}{5, 53, 60}
\definecolor{mydarkdarkturquoise}{RGB}{8, 93, 110}
\definecolor{mydarkturquoise}{RGB}{28, 143, 150}
\definecolor{mypaleturquoise}{RGB}{230, 255, 255}
\definecolor{myturquoise}{RGB}{48, 213, 200}

\definecolor{mygreen}{RGB}{0, 200, 0}
\definecolor{mydarkgreen}{RGB}{0, 120, 0}
\definecolor{mygreen2}{RGB}{245, 255, 230}

\definecolor{mygrey}{RGB}{120, 120, 120}
\definecolor{mypalegrey}{RGB}{160, 160, 160}
\definecolor{mydarkgrey}{RGB}{80, 80, 160}

\definecolor{mydarkred}{RGB}{160, 30, 30}
\definecolor{mylightred}{RGB}{255, 150, 150}
\definecolor{myred}{RGB}{200, 110, 110}
\definecolor{myblackred}{RGB}{120, 40, 40}


\definecolor{myblackmaroon}{RGB}{50, 0, 15}

\definecolor{mygreen}{RGB}{0, 200, 0}
\definecolor{mygreen2}{RGB}{205, 255, 200}

\definecolor{mydarkcolor}{RGB}{60, 25, 155}
\definecolor{mylightcolor}{RGB}{130, 180, 250}

\setbeamertemplate{itemize items}[default]

\setbeamertemplate{itemize item}{\color{myblackmaroon}$\blacksquare$}
\setbeamertemplate{itemize subitem}{\color{mydarkdarkturquoise}$\blacktriangleright$}
\setbeamertemplate{itemize subsubitem}{\color{mygray}$\blacksquare$}

\setbeamercolor{palette quaternary}{fg=white,bg=myblackmaroon}
\setbeamercolor{titlelike}{parent=palette quaternary}

\setbeamercolor{palette quaternary2}{fg=black,bg=mypaleblue}
\setbeamercolor{frametitle}{parent=palette quaternary2}

\setbeamerfont{frametitle}{size=\Large,series=\scshape}
\setbeamerfont{framesubtitle}{size=\normalsize,series=\upshape}





%%%%%%%%%%%%%%%%%%%%%%%%%%%%
% block settings

\setbeamercolor{block title}{bg=red!30,fg=black}

\setbeamercolor*{block title example}{bg=mygreen!40!white,fg=black}

\setbeamercolor*{block body example}{fg= black, bg= mygreen2}


%%%%%%%%%%%%%%%%%%%%%%%%%%%%
% URL settings
\hypersetup{
	colorlinks=true,
	linkcolor=blue,
	filecolor=blue,      
	urlcolor=blue,
}

%%%%%%%%%%%%%%%%%%%%%%%%%%

\renewcommand{\familydefault}{\rmdefault}

\usepackage{amsmath}
\usepackage{mathtools}

\usepackage{subcaption}

\usepackage{qrcode}

\DeclareMathOperator*{\argmin}{arg\,min}
\newcommand{\bo}[1] {\mathbf{#1}}

\newcommand{\R}{\mathbb{R}} 
\newcommand{\T}{^\top}     



\newcommand{\mydate}{Fall 2023}

\newcommand{\mygit}{\textcolor{blue}{\href{https://github.com/SergeiSa/Mechatronics-2023}{github.com/SergeiSa/Mechatronics-2023}}}

\newcommand{\myqr}{ \textcolor{black}{\qrcode[height=1.5in]{https://github.com/SergeiSa/Mechatronics-2023}}
}

\newcommand{\myqrframe}{
	\begin{frame}
		\centerline{Lecture slides are available via Github, links are on Moodle}
		\bigskip
		\centerline{You can help improve these slides at:}
		\centerline{\mygit}
		\bigskip
		\myqr
	\end{frame}
}


\newcommand{\bref}[2] {\textcolor{blue}{\href{#1}{#2}}}

%%%%%%%%%%%%%%%%%%%%%%%%%%%%
% code settings

\usepackage{listings}
\usepackage{color}
% \definecolor{mygreen}{rgb}{0,0.6,0}
% \definecolor{mygray}{rgb}{0.5,0.5,0.5}
\definecolor{mymauve}{rgb}{0.58,0,0.82}
\lstset{ 
	backgroundcolor=\color{white},   % choose the background color; you must add \usepackage{color} or \usepackage{xcolor}; should come as last argument
	basicstyle=\footnotesize,        % the size of the fonts that are used for the code
	breakatwhitespace=false,         % sets if automatic breaks should only happen at whitespace
	breaklines=true,                 % sets automatic line breaking
	captionpos=b,                    % sets the caption-position to bottom
	commentstyle=\color{mygreen},    % comment style
	deletekeywords={...},            % if you want to delete keywords from the given language
	escapeinside={\%*}{*)},          % if you want to add LaTeX within your code
	extendedchars=true,              % lets you use non-ASCII characters; for 8-bits encodings only, does not work with UTF-8
	firstnumber=0000,                % start line enumeration with line 0000
	frame=single,	                   % adds a frame around the code
	keepspaces=true,                 % keeps spaces in text, useful for keeping indentation of code (possibly needs columns=flexible)
	keywordstyle=\color{blue},       % keyword style
	language=Octave,                 % the language of the code
	morekeywords={*,...},            % if you want to add more keywords to the set
	numbers=left,                    % where to put the line-numbers; possible values are (none, left, right)
	numbersep=5pt,                   % how far the line-numbers are from the code
	numberstyle=\tiny\color{mygray}, % the style that is used for the line-numbers
	rulecolor=\color{black},         % if not set, the frame-color may be changed on line-breaks within not-black text (e.g. comments (green here))
	showspaces=false,                % show spaces everywhere adding particular underscores; it overrides 'showstringspaces'
	showstringspaces=false,          % underline spaces within strings only
	showtabs=false,                  % show tabs within strings adding particular underscores
	stepnumber=2,                    % the step between two line-numbers. If it's 1, each line will be numbered
	stringstyle=\color{mymauve},     % string literal style
	tabsize=2,	                   % sets default tabsize to 2 spaces
	title=\lstname                   % show the filename of files included with \lstinputlisting; also try caption instead of title
}


%%%%%%%%%%%%%%%%%%%%%%%%%%%%
% URL settings
\hypersetup{
	colorlinks=false,
	linkcolor=blue,
	filecolor=blue,      
	urlcolor=blue,
}

%%%%%%%%%%%%%%%%%%%%%%%%%%

%%%%%%%%%%%%%%%%%%%%%%%%%%%%
% tikz settings

\usepackage{tikz}
\tikzset{every picture/.style={line width=0.75pt}}


\title{ODEs and DAEs}
\subtitle{Contact-aware Control, Lecture 1}
\author{by Sergei Savin}
\centering
\date{\mydate}



\begin{document}
\maketitle


\begin{frame}{Content}

\begin{itemize}
\item Motivation
\end{itemize}

\end{frame}




\begin{frame}{Ordinary Differential Equations}
% \framesubtitle{Parameter estimation}
\begin{flushleft}

A general form of an \emph{n-th order ordinary differential equation} (ODE) is:

\begin{equation}
	F \left( \frac{d^n x}{dt^n}, ..., \frac{d x}{dt}, x, t  \right) = 0
\end{equation}

\bigskip

Notice that $x(t)$ is a scalar variable. A normal form of an ODE is:

\begin{equation}
	\frac{d^n x}{dt^n} = f \left( \frac{d^{n-1} x}{dt^{n-1}}, ..., \frac{d x}{dt}, x, t  \right)
\end{equation}

\end{flushleft}
\end{frame}




\begin{frame}{Ordinary Differential Equations - examples}
	% \framesubtitle{Parameter estimation}
	\begin{flushleft}
		
		Below we see two examples of an ODE in the general form:
		
		\begin{equation}
			\dot x + \sin (x+1) = \sin(2t)
		\end{equation}
		
		\begin{equation}
			\dot x^3 + x = 0
		\end{equation}
	
	...and two examples of ODEs in the normal form:
		
		\begin{equation}
			\dot x = \sin(2t) - \sin (x+1) 
		\end{equation}
		
		\begin{equation}
			\ddot x = -5\dot x - 2 x
		\end{equation}
		
	\end{flushleft}
\end{frame}



\begin{frame}{Linear ODE}
	% \framesubtitle{Parameter estimation}
	\begin{flushleft}
		
		A general form of a \emph{linear} n-th order ODE is:
		
		\begin{equation}
			a_n \frac{d^n x}{dt^n} + ... + a_1 \frac{d x}{dt} + a_0 x  = f(t)
		\end{equation}
		
		\bigskip
		
		Note that it is trivial to transform general form linear ODE to the normal form:
		
		\begin{equation}
			\frac{d^n x}{dt^n} = -\frac{a_{n-1}}{a_n } \frac{d^{n-1} x}{dt^{n-1}} - ... - \frac{a_1}{a_n }  \frac{d x}{dt} - \frac{a_0}{a_n } x  + \frac{1}{a_n }f(t)
		\end{equation}
	
	\end{flushleft}
\end{frame}




\begin{frame}{ODE matrix form}
	% \framesubtitle{Parameter estimation}
	\begin{flushleft}
		
		A general form of a first order matrix ODE is:
		
		\begin{equation}
			\label{eq:matrix_ODE_general_form}
			\bo{F}(\dot{\bo{x}}, \bo{x}, t) = 0
		\end{equation}
		
		\bigskip
		
		Normal form is:
		
		\begin{equation}
			\dot{\bo{x}} = \bo{f}(\bo{x}, t)
		\end{equation}
	
		If the ODE \eqref{eq:matrix_ODE_general_form} is linear with respect $\dot{\bo{x}}$ we could transform it into the normal form:
		
		\begin{equation}
			\dot{\bo{x}} = - \bo{M}^{-1} \bo{F}(0, \bo{x}, t) 
		\end{equation}
		%
		where $ \bo{M} = d\bo{F} / d\dot{\bo{x}}$.
	
	\end{flushleft}
\end{frame}




\begin{frame}{DAE matrix form}
	% \framesubtitle{Parameter estimation}
	\begin{flushleft}
		
		A general form of a first order matrix \emph{differential-algebraic equations} (DAE) is:
		
		\begin{equation}
			\bo{F}(\dot{\bo{x}}, \bo{x}, t) = 0
		\end{equation}
		
		where $\bo{M} = d\bo{F} / d\dot{\bo{x}}$ and it is either rectangular or $\text{det} (\bo{M}) = 0$.
		
		\bigskip
		
		\textcolor{mygrey}{
		Let us observe how similar are DAE and general form ODEs.}
		
	\end{flushleft}
\end{frame}



\begin{frame}{DAE examples}
	% \framesubtitle{Parameter estimation}
	\begin{flushleft}
		
		DAE examples:
		
		\begin{align}
		\begin{cases}
						\dot x_1 + \dot x_2 + x_1 x_2 +  \sin(t) = 0\\
						x_1 - x_2 = 0
		\end{cases}
		\end{align}
		
		\begin{align}
			\begin{cases}
				\dot x_1 + \dot x_2 - 7 = 0\\
				\dot x_1 + \dot x_2 + x_1 + \cos(x_2) = 0
			\end{cases}
		\end{align}
		
		
	\end{flushleft}
\end{frame}



\begin{frame}{DAE with explicit algebraic variables}
	% \framesubtitle{Parameter estimation}
	\begin{flushleft}
		
		An normal form DAE with algebraic variables is:
		
		\begin{equation}
			\begin{cases}
				\dot{\bo{x}} = \bo{f}(\bo{x}, \lambda, t) \\
				\bo{g}(\bo{x}, t) = 0
			\end{cases}
		\end{equation}
	
		In this equation, $\lambda$ are algebraic variables, and $\bo{g}(\bo{x}, t) = 0$ are \emph{constraints}.
		
		\bigskip
		
		\textcolor{mygrey}{
		Nothing prevents us from defining constraints as $\bo{g}(\dot{\bo{x}}, \bo{x}, t) = 0$, but the clarity of the definition will be lost.
		}

	\end{flushleft}
\end{frame}


\begin{frame}{DAE linear in algebraic variables, 1}
	% \framesubtitle{Parameter estimation}
	\begin{flushleft}
		
		A DAE linear in algebraic variables is:
		
		\begin{equation}
			\begin{cases}
				\dot{\bo{x}} = \bo{f}(\bo{x}, t) + \bo{H}(\bo{x}, t) \lambda \\
				\bo{g}(\bo{x}, t) = 0
			\end{cases}
		\end{equation}
		
		We could solve it by differentiating constraints and expressing $\dot{\bo{x}}$ and $\lambda$:
		
		\begin{equation}
		\begin{cases}
			\dot{\bo{x}} = \bo{f}(\bo{x}, t) + \bo{H}(\bo{x}, t) \lambda \\
			\bo{G}(\bo{x}, t)\dot{\bo{x}} + \bo{g}_0(\bo{x}, t)  = 0
		\end{cases}
		\end{equation}
	
		where $\bo{G} = d\bo{g} / d\bo{x}$ and $\frac{d}{dt} \bo{g}(\bo{x}, t) - \bo{G}\dot{\bo{x}}$.
		
	\end{flushleft}
\end{frame}



\begin{frame}{DAE linear in algebraic variables, 2}
	% \framesubtitle{Parameter estimation}
	\begin{flushleft}
		
		DAE can be re-written as:
		
		\begin{equation}
			\begin{bmatrix}
				\bo{I} & -\bo{H}(\bo{x}, t) \\
				\bo{G}(\bo{x}, t) & 0
			\end{bmatrix}
		\begin{bmatrix}
			\dot{\bo{x}} \\ \lambda
		\end{bmatrix}
		=
		\begin{bmatrix}
			\bo{f}(\bo{x}, t) \\
			-\bo{g}_0(\bo{x}, t)
		\end{bmatrix}
		\end{equation}
		
		Solving this system we can find $\dot{\bo{x}}$ and integrate it forward.
		
	\end{flushleft}
\end{frame}



\begin{frame}{DAE linear in algebraic variables, 3}
	% \framesubtitle{Parameter estimation}
	\begin{flushleft}
		
		We could propose a more general form of the DAE linear in derivatives and algebraic variables:
		
		\begin{equation}
			\begin{bmatrix}
				\bo{M}(\bo{x}, t) & -\bo{H}(\bo{x}, t) \\
				\bo{G}(\bo{x}, t) & 0
			\end{bmatrix}
			\begin{bmatrix}
				\dot{\bo{x}} \\ \lambda
			\end{bmatrix}
			=
			\begin{bmatrix}
				\bo{f}_0(\bo{x}, t) \\
				-\bo{g}_0(\bo{x}, t)
			\end{bmatrix}
		\end{equation}
		
		where $\text{det} (\bo{M}) \neq 0$. 
		
		\bigskip
		
		We will find these types of systems in robotics. Inclusion of inertia matrix $\bo{M}$ does not change the process of solving the equation.
		
	\end{flushleft}
\end{frame}


\myqrframe

\end{document}
